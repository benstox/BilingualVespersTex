% TONUS ORATIONIS

\section*{Tonus Orationis}
\label{sec:TonusOrationis}

\chead{\color{benred8}\textsc{Tonus Orationis}}
\thispagestyle{plain}

\gregorioscore[a]{LatinVersion/TonusOrationis1.gtex}

\vspace{1mm}

\begin{rubric}
In Orationibus fit flexa tantum in fine prim\ae\ distin\libertineGlyph{c_t}ionis ( sive \GreDagger\ sive \GreStar\ ) :

\end{rubric}

\gregorioscore[a]{LatinVersion/TonusOrationis3.gtex}

\vspace{1mm}

\begin{rubric}
Si autem Oratio sit solito longior, flexa fieri pote\libertineGlyph{s_t} in ipso corpore Orationis semel vel pluries, prout fert textus.

\end{rubric}

\subsection*{Ali\ae\ Conclusiones.}

\gregorioscore[a]{LatinVersion/TonusOrationis2.gtex}

\vspace*{-1.0mm}

\subsection*{Tonus Simplex.}

\label{OratioSimplex}

\begin{rubric}
Ad Completorium et ad Antiphonas de B. M. V. in fine Officii, servatur sequens tonus :

\end{rubric}

{
\grechangedim{beforeinitialshift}{0.25mm}{scalable}
\grechangedim{afterinitialshift}{0.25mm}{scalable}
\gregorioscore[a]{LatinVersion/OratioSimplex1.gtex}

}

\subsection*{Alia Conclusio Toni Simplicis.}

\gregorioscore[a]{LatinVersion/OratioSimplex2.gtex}

\newpage

% COMMEMORATIONS

\section*{Commemorationes}
\label{sec:Commem}

\chead{\color{benred8}\textsc{Commemorationes}}
\thispagestyle{plain}

% COMMEMORATION OF THE CROSS

\subsection*{Commemoratio de Cruce.}

\begin{rubric}
In Feriali tantum Officio, extra Tempus Paschale, etiam quando non dicuntur Preces :

\end{rubric}

\greannotation{II}
\gregorioscore[a]{LatinVersion/CommemoratioCrucis.gtex}

\begin{rubric}
Sequens tonus adhibetur in omnibus versiculis Commemorationum :

\end{rubric}

{\gresetinitiallines{0}
\gregorioscore[a]{LatinVersion/OmnisTerraV.gtex}

}

\begin{response}
\hspace{1.0 mm}\Rbar\hspace{0.3mm} Psalmum \hspace{3.4mm} dicat \hspace{3.4mm} n\'{o}mini \hspace{3.9mm} tuo \hspace{3.4mm} D\'{o}mine.

\end{response}

\subsection*{\textcolor{black}{Or\'{e}mus.}\capitulumSpace \emph{Oratio.}}

\begin{response}\lettrine[nindent=-1.5mm]{P}{e}rp\'{e}tua nos, qu\'{\ae}sumus D\'{o}mine, pace cu\libertineGlyph{s_t}\'{o}di : \GreStar\ quos per lignum sanct\ae\ Crucis red\'{i}mere dign\'{a}tus es.

\end{response}

\begin{rubric}
Sequentes Commemorationes de San\libertineGlyph{c_t}a Maria, de San\libertineGlyph{c_t}o Joseph, de Apo\libertineGlyph{s_t}olis, de Patrono Ecclesi\ae , et de Pace canuntur ab O\libertineGlyph{c_t}ava Penteco\libertineGlyph{s_t}es usque ad Adventum, et ab O\libertineGlyph{c_t}ava Epiphani\ae\ usque ad Dominicam Passionis, exceptis Officiis Duplicibus et infra O\libertineGlyph{c_t}avas. Quando dicitur Officium Beat\ae\ Mari\ae\ Virginis, non fit alia Commemoratio de ea.

\end{rubric}

% COMMEMORATION OF THE B. V. M.

\subsection*{Commemoratio de San\libertineGlyph{c_t}a Maria.}

\greannotation{I}
\gregorioscore[a]{LatinVersion/CommemoratioBMV.gtex}

\begin{response}
\Vbar\ Ora pro nobis, san\libertineGlyph{c_t}a Dei G\'{e}netrix.

\Rbar\ Ut digni effici\'{a}mur promissi\'{o}nibus Chri\libertineGlyph{s_t}i.

\end{response}

\subsection*{\textcolor{black}{Or\'{e}mus.}\capitulumSpace \emph{Oratio.}}

\begin{response}\lettrine{C}{o}nc\'{e}de nos f\'{a}mulos tuos, qu\'{\ae}sumus D\'{o}mine Deus, perp\'{e}tua mentis et c\'{o}rporis sanit\'{a}te gaud\'{e}re, \GreStar\ et glori\'{o}sa be\'{a}t\ae\ Mar\'{i}\ae\ semper V\'{i}rginis intercessi\'{o}ne, a pr\ae s\'{e}nti liber\'{a}ri tri\libertineGlyph{s_t}\'{i}tia, et \ae t\'{e}rna p\'{e}rfrui l\ae t\'{i}tia.

\end{response}

\begin{rubric}
Ab Octava Epiphani\ae\ usque ad Purificationem :

\end{rubric}

\begin{response}
\Vbar\ Po\libertineGlyph{s_t} partum Virgo inviol\'{a}ta permans\'{i}\libertineGlyph{s_t}i.

\Rbar\ Dei G\'{e}netrix, interc\'{e}de pro nobis.

\end{response}

\subsection*{\textcolor{black}{Or\'{e}mus.}\capitulumSpace \emph{Oratio.}}

\begin{response}\lettrine{D}{e}us, qui sal\'{u}tis \ae t\'{e}rn\ae , be\'{a}t\ae\ Mar\'{i}\ae\ virginit\'{a}te fec\'{u}nda, hum\'{a}no g\'{e}neri pr\'{\ae}mia \mbox{pr\ae\libertineGlyph{s_t}it\'{i}\libertineGlyph{s_t}i : \GreDagger} tr\'{i}bue, qu\'{\ae}sumus ; ut ipsam pro nobis interc\'{e}dere senti\'{a}mus, \GreStar\ per quam mer\'{u}imus au\libertineGlyph{c_t}\'{o}rem vit\ae\ susc\'{i}pere, D\'{o}minum no\libertineGlyph{s_t}rum Jesum Chri\libertineGlyph{s_t}um F\'{i}lium tuum.

\end{response}

%% COMMEMORATION OF ST JOSEPH

\subsection*{Commemoratio de San\libertineGlyph{c_t}o Joseph.}

\greannotation{VIII}
\gregorioscore[a]{LatinVersion/CommemoratioJoseph.gtex}

\begin{response}
\Vbar\ Gl\'{o}ria et div\'{i}ti\ae\ in domo ejus.

\Rbar\ Et ju\libertineGlyph{s_t}\'{i}tia ejus manet in s\'{\ae}culum s\'{\ae}culi.

\end{response}

\subsection*{\textcolor{black}{Or\'{e}mus.}\capitulumSpace \emph{Oratio.}}

\begin{response}\lettrine{D}{e}us, qui ineff\'{a}bili provid\'{e}ntia be\'{a}tum Joseph san\libertineGlyph{c_t}\'{i}ssim\ae\ Genetr\'{i}cis tu\ae\ sponsum el\'{i}gere dign\'{a}tus es : \GreStar\ pr\ae \libertineGlyph{s_t}a, qu\'{\ae}sumus ; ut quem prote\libertineGlyph{c_t}\'{o}rem vener\'{a}mur in terris, intercess\'{o}rem hab\'{e}re mere\'{a}mur in c\ae lis.

\end{response}

%% COMMEMORATION OF THE APOSTLES

\subsection*{Commemoratio de Apo\libertineGlyph{s_t}olis.}

\greannotation{VIII}
\gregorioscore[a]{LatinVersion/CommemoratioApost.gtex}

\begin{response}
\Vbar\ Con\libertineGlyph{s_t}\'{i}tues eos pr\'{i}ncipes super omnem terram.

\Rbar\ M\'{e}mores erunt n\'{o}minis tui D\'{o}mine.

\end{response}

\subsection*{\textcolor{black}{Or\'{e}mus.}\capitulumSpace \emph{Oratio.}}

\begin{response}\lettrine{D}{e}us, cujus d\'{e}xtera be\'{a}tum Petrum ambul\'{a}ntem in fl\'{u}\libertineGlyph{c_t}ibus, ne merger\'{e}tur, er\'{e}xit, et coap\'{o}\libertineGlyph{s_t}olum ejus Paulum t\'{e}rtio naufrag\'{a}ntem de prof\'{u}ndo p\'{e}lagi liber\'{a}vit : \GreStar\ ex\'{a}udi nos prop\'{i}tius, et conc\'{e}de ; ut amb\'{o}rum m\'{e}ritis \ae ternit\'{a}tis gl\'{o}riam consequ\'{a}mur.

\end{response}

% so that the title isn't stuck on the previous page:
\newpage

%% PATRONAL COMMEMORATION

\subsection*{Commemoratio de Patrono vel Titulari Ecclesi\ae .}

\begin{rubric}
Fiat Commemoratio consueta ante vel po\libertineGlyph{s_t} Commemorationes pr\ae di\libertineGlyph{c_t}as pro dignitate\\illius, e.~g.~:

\end{rubric}

\subsection*{Commemoratio de San\libertineGlyph{c_t}o Bed\ae .}

\greannotation{II}
\gregorioscore[a]{LatinVersion/CommemoratioBedae.gtex}

\begin{response}
\Vbar\ Am\'{a}vit eum D\'{o}minus, et orn\'{a}vit eum.

\Rbar\ Stolam gl\'{o}ri\ae\ \'{i}nduit eum.

\end{response}

\subsection*{\textcolor{black}{Or\'{e}mus.}\capitulumSpace \emph{Oratio.}}

\begin{response}\lettrine{D}{e}us, qui Eccl\'{e}siam tuam be\'{a}ti Bed\ae\ Confess\'{o}ris tui atque Do\libertineGlyph{c_t}\'{o}ris eruditi\'{o}ne clar\'{i}ficas~:~\GreDagger\ conc\'{e}de prop\'{i}tius f\'{a}mulis tuis, \GreStar\ ejus semper illustr\'{a}ri sapi\'{e}ntia et m\'{e}ritis adjuv\'{a}ri.

\end{response}

\begin{rubric}
Ultimo loco~:

\end{rubric}


%% COMMEMORATION OF PEACE

\subsection*{Commemoratio de Pace.}

\greannotation{II}
\gregorioscore[a]{LatinVersion/CommemoratioPacis.gtex}

\begin{response}
\Vbar\ Fiat pax in virt\'{u}te tua.

\Rbar\ Et abund\'{a}ntia in t\'{u}rribus tuis.

\end{response}

\subsection*{\textcolor{black}{Or\'{e}mus.}\capitulumSpace \emph{Oratio.}}

\begin{response}\lettrine{D}{e}us, a quo san\libertineGlyph{c_t}a desid\'{e}ria, re\libertineGlyph{c_t}a cons\'{i}lia, et justa sunt \'{o}pera : \GreDagger\ da servis tuis illam, quam mundus dare non pote\libertineGlyph{s_t}, pacem ; \GreStar\ ut et corda no\libertineGlyph{s_t}ra mand\'{a}tis tuis d\'{e}dita, et h\'{o}\libertineGlyph{s_t}ium subl\'{a}ta form\'{i}dine, t\'{e}mpora sint tua prote\libertineGlyph{c_t}ione tranqu\'{i}lla. Per D\'{o}minum no\libertineGlyph{s_t}rum Jesum Chri\libertineGlyph{s_t}um F\'{i}lium tuum : qui tecum vivit et regnat in unit\'{a}te Sp\'{i}ritus San\libertineGlyph{c_t}i Deus, per \'{o}mnia s\'{\ae}cula s\ae cul\'{o}rum. \Rbar\ Amen.

\end{response}

\newpage

% SOLEMN MARIAN ANTIPHONS

\section*{Antiphon\ae\ Finales B. Mari\ae\ Virginis}
\label{sec:AntBMV}

\chead{\color{benred8}\textsc{Antiphon\ae\ Finales B. M. V.}}
\thispagestyle{plain}

% ALMA REDEMPTORIS MATER SOLEMN

\begin{rubric}
\hspace*{25pt}A Vesperis Sabbati ante Dominicam I Adventus usque ad II Vesperas Purificationis, diei Februarii inclusive :

\end{rubric}

\greannotation{V}
\gregorioscore[a]{LatinVersion/AlmaRedemptorisSolemn.gtex}

\vspace{2mm}

\begin{rubric}
In Adventu :

\end{rubric}

\begin{response}
\Vbar\ Angelus D\'{o}mini nunti\'{a}vit Mar\'{i}\ae .

\Rbar\ Et conc\'{e}pit de Sp\'{i}ritu San\libertineGlyph{c_t}o.

\end{response}

\subsection*{\textcolor{black}{Or\'{e}mus.}\capitulumSpace \emph{Oratio.}}

\begin{response}\lettrine{G}{r}\'{a}tiam tuam, qu\'{\ae}sumus D\'{o}mine, m\'{e}ntibus no\libertineGlyph{s_t}ris inf\'{u}nde : \GreDagger\ ut qui, Angelo nunti\'{a}nte, Chri\libertineGlyph{s_t}i F\'{i}lii tui Incarnati\'{o}nem cogn\'{o}vimus, \GreStar\ per passi\'{o}nem ejus et crucem ad resurre\libertineGlyph{c_t}i\'{o}nis gl\'{o}riam perduc\'{a}mur. Per e\'{u}mdem Chri\libertineGlyph{s_t}um D\'{o}minum no\libertineGlyph{s_t}rum. \Rbar\ Amen.

\end{response}

\begin{rubric}
A primis Vesperis Nativitatis Domini et deinceps :

\end{rubric}

\begin{response}
\Vbar\ Po\libertineGlyph{s_t} partum Virgo inviol\'{a}ta permans\'{i}\libertineGlyph{s_t}i.

\Rbar\ Dei G\'{e}netrix, interc\'{e}de pro nobis.

\end{response}

\subsection*{\textcolor{black}{Or\'{e}mus.}\capitulumSpace \emph{Oratio.}}

\begin{response}\lettrine{D}{e}us, qui sal\'{u}tis \ae t\'{e}rn\ae , be\'{a}t\ae\ Mar\'{i}\ae\ virginit\'{a}te fec\'{u}nda, hum\'{a}no g\'{e}neri pr\'{\ae}mia \mbox{pr\ae\libertineGlyph{s_t}it\'{i}\libertineGlyph{s_t}i : \GreDagger} tr\'{i}bue, qu\'{\ae}sumus ; ut ipsam pro nobis interc\'{e}dere senti\'{a}mus, \GreStar\ per quam mer\'{u}imus au\libertineGlyph{c_t}\'{o}rem vit\ae\ susc\'{i}pere, D\'{o}minum no\libertineGlyph{s_t}rum Jesum Chri\libertineGlyph{s_t}um F\'{i}lium tuum. \Rbar\ Amen.

\end{response}

% AVE REGINA CAELORUM SOLEMN

%newpage keeps the rubric from remaining alone on the Alma Redemptoris Mater page
\newpage

\begin{rubric}Po\libertineGlyph{s_t} Purificationem, id est, a fine Completorii illius diei 2 Februarii inclusive, etiam quando transferatur fe\libertineGlyph{s_t}um Purificationis B. M. V., usque ad Feriam V in C\oe na Domini exclusive.\end{rubric}

\greannotation{VI}
\gregorioscore[a]{LatinVersion/AveReginaCaelorumSolemn.gtex}

\vspace{2mm}

\begin{response}
\Vbar\ Dign\'{a}re me laud\'{a}re te, Virgo sacr\'{a}ta.

\Rbar\ Da mihi virt\'{u}tem contra hostes tuos.

\end{response}

\subsection*{\textcolor{black}{Or\'{e}mus.}\capitulumSpace \emph{Oratio.}}

\begin{response}\lettrine{C}{o}nc\'{e}de, mis\'{e}ricors Deus, fragilit\'{a}ti no\libertineGlyph{s_t}r\ae\ pr\ae s\'{i}dium : \GreDagger\ ut qui san\libertineGlyph{c_t}\ae\ Dei Genitr\'{i}cis mem\'{o}riam \'{a}gimus, \GreStar\ intercessi\'{o}nis ejus aux\'{i}lio, a no\libertineGlyph{s_t}ris iniquit\'{a}tibus resurg\'{a}mus. Per e\'{u}mdem Chri\libertineGlyph{s_t}um D\'{o}minum no\libertineGlyph{s_t}rum. \Rbar\ Amen.

\end{response}

% REGINA CAELI SOLEMN

\subsection*{}

\begin{rubric}
A Completorio Sabbati San\libertineGlyph{c_t}i usque ad Nonam Sabbati infra O\libertineGlyph{c_t}avam Penteco\libertineGlyph{s_t}es inclusive.

\end{rubric}

\greannotation{VI}
\gregorioscore[a]{LatinVersion/ReginaCaeliSolemn.gtex}

\vspace{2mm}

\begin{response}
\Vbar\ Gaude et l\ae t\'{a}re Virgo Mar\'{i}a, allel\'{u}ia.

\Rbar\ Quia surr\'{e}xit D\'{o}minus vere, allel\'{u}ia.

\end{response}

\subsection*{\textcolor{black}{Or\'{e}mus.}\capitulumSpace \emph{Oratio.}}

\begin{response}\lettrine{D}{e}us, qui per resurre\libertineGlyph{c_t}i\'{o}nem F\'{i}lii tui D\'{o}mini no\libertineGlyph{s_t}ri Jesu Chri\libertineGlyph{s_t}i mundum l\ae tific\'{a}re dign\'{a}tus es : \GreDagger\ pr\ae\libertineGlyph{s_t}a, qu\'{\ae}sumus ; ut per ejus Genitr\'{i}cem V\'{i}rginem Mar\'{i}am, \GreStar\ perp\'{e}tu\ae\ capi\'{a}mus g\'{a}udia vit\ae . Per e\'{u}mdem Chri\libertineGlyph{s_t}um D\'{o}minum no\libertineGlyph{s_t}rum. \Rbar\ Amen.

\end{response}

% SALVE REGINA SOLEMN

\subsection*{}

\begin{rubric}
A primis Vesperis Fe\libertineGlyph{s_t}i Ss.~Trinitatis usque ad Nonam Sabbati ante Adventum inclusive.

\end{rubric}

{
\greannotation{I}
\gregorioscore[a]{LatinVersion/SalveReginaSolemn.gtex}

}

\vspace{2mm}

\begin{response}
\Vbar\ Ora pro nobis, san\libertineGlyph{c_t}a Dei G\'{e}netrix.

\Rbar\ Ut digni effici\'{a}mur promissi\'{o}nibus Chri\libertineGlyph{s_t}i.

\end{response}

\subsection*{\textcolor{black}{Or\'{e}mus.}\capitulumSpace \emph{Oratio.}}

\begin{response}\lettrine{O}{m}n\'{i}potens sempit\'{e}rn\ae\ Deus, qui glori\'{o}s\ae\ V\'{i}rginis Matris Mar\'{i}\ae\ corpus et \'{a}nimam, ut dignum F\'{i}lii tui habit\'{a}culum \'{e}ffici merer\'{e}tur, Sp\'{i}ritu San\libertineGlyph{c_t}o cooper\'{a}nte pr\ae par\'{a}\libertineGlyph{s_t}i : \GreDagger\ da, ut cujus commemorati\'{o}ne l\ae t\'{a}mur, \GreStar\ ejus pia intercessi\'{o}ne ab in\libertineGlyph{s_t}\'{a}ntibus malis et a morte perp\'{e}tua liber\'{e}mur. Per e\'{u}mdem Chri\libertineGlyph{s_t}um D\'{o}minum no\libertineGlyph{s_t}rum. \Rbar\ Amen.

\end{response}

\newpage

% SIMPLE MARIAN ANTIPHONS

\section*{Antiphon\ae\ Finales B. Mari\ae\ Virginis\\\begin{Large}in Cantu Simplici\end{Large}}

\thispagestyle{plain}

{
\greannotation{V}
\gregorioscore[a]{LatinVersion/AlmaRedemptorisSimple.gtex}

}

\vspace{2mm}

\greannotation{VI}
\gregorioscore[a]{LatinVersion/AveReginaCaelorumSimple.gtex}

\vspace{2mm}

\greannotation{VI}
\gregorioscore[a]{LatinVersion/ReginaCaeliSimple.gtex}

\vspace{2mm}

\greannotation{V}
\gregorioscore[a]{LatinVersion/SalveReginaSimple.gtex}

\vspace{2mm}

\newpage

% GUIDE TO SINGING IN ENGLISH

\section*{Chanting the Psalms in English}

\thispagestyle{plain}

\begin{response}
The English Psalm Tones used here are based as closely as possible on the Latin ones. The structure of the English language means that some modifications must be made however. In brief, each tone has a certain number of cadential notes or note groups at the middle and end of each verse. For example, Psalm 113 is sung to the Peregrine Tone. I have adapted the Latin Peregrine Tone into a six-note cadence in the middle of the verse and a four-note cadence at the end. The tenor notes have been separated from the cadences by a \pipe\ symbol. Note that the cadences may begin with the same note as the preceding tenor note, which is what happens in both half-verses here. Notice also that the last ``note'' of the end cadence is actually a sequence of two notes:

\end{response}

\gregorioscore[a]{EnglishVersion/ToneExplanation01.gtex}

\begin{response}
Bold text represents a syllable that uses up two notes of the cadence. Underlined text represents more than one syllable that share a single note of the cadence:

\end{response}

\gregorioscore[a]{EnglishVersion/ToneExplanation02.gtex}

\begin{response}
Bold and underlined text can overlap, as well:

\end{response}

\gregorioscore[a]{EnglishVersion/ToneExplanation03.gtex}

\newpage

% DEDICATION PAGE

\thispagestyle{empty}

\topskip0pt

\vspace*{\fill}

\begin{center}
\greseparator{5}{36}

\vspace{2 mm}

Amice qui legis,

De caritate tua

Ora pro animabus

\begin{Large}
Cornelii et Margareth\ae\ Sto\libertineGlyph{c_k}ermans
\end{Large}

atque

\begin{Large}
Aurelii et Angel\ae\ Horatii.
\end{Large}

Requiem \ae ternam dona eis Domine,

Et lux perpetua luceat eis.

Requiescant in pace.

Amen.

\greseparator{5}{36}
\end{center}

\vspace*{\fill}
