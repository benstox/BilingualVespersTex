% DOMINICA AD VESPERAS
% START PARALLEL TEXTS
\begin{pages}
\begin{Leftside}
\firstlinenum{10000}\linenumincrement{10000}\beginnumbering\pstart

\begin{center}\begin{Huge}\textsc{\textcolor{benred8}{Dominica ad Vesperas}}\end{Huge}\end{center}

\pend\pstart

\vspace{2mm}

\begin{center}Pater no\libertineGlyph{s_t}er. Ave Mar\'{i}a.\end{center}

\pend\pstart

\grechangestyle{initial}{\fontsize{43}{43}\selectfont}
\greannotation{\upshape\Vbar}
\gregorioscore[a]{LatinVersion/DeusInAdjutorium.gtex}

\pend\pstart

\begin{rubric}
A Septuagesima usque ad Pascha, loco \emph{\textcolor{black}{Allelúia}}, canitur :

\end{rubric}

\gregorioscore[a]{LatinVersion/LausTibi.gtex}

\pend\pstart

\vspace{2mm}

\begin{center}\begin{large}\textbf{\textsc{\textcolor{benred8}{1 Antiphona. VII c 2}}}\end{large}\end{center}

\vspace{2mm}

\pend\pstart

{\gresetinitiallines{2}
\grechangestyle{initial}{\fontsize{160}{160}\selectfont}
\gregorioscore[a]{LatinVersion/Dominica/DomAnt1-01.gtex}

}

\pend\pstart

\redTitle{Psalmus 109.}

\pend\pstart

\begin{psalmtext}
% Donec ponam ini\textbf{mí}cos \textbf{tu}os, \GreStar\ scabéllum \textbf{pe}dum tu\textbf{ó}rum.

Virgam virtútis tuæ emíttet Dómi\textbf{nus} ex \textbf{Si}on : \GreStar\ domináre in médio inimi\textbf{có}rum tu\textbf{ó}rum.

Tecum princípium in die virtútis tuæ in splendóri\textbf{bus} sanc\textbf{tó}rum : \GreStar\ ex útero ante lucíferum \textbf{gé}nu\textbf{i} te.

Jurávit Dóminus et non pæni\textbf{té}bit \textbf{e}um : \GreStar\ Tu es sacérdos in æt\'{e}rnum secúndum órdi\textbf{nem} Mel\textbf{chí}sedech.

Dóminus a \textbf{dex}tris \textbf{tu}is, \GreStar\ confrégit in die iræ \textbf{su}æ \textbf{re}ges.

Judicábit in natiónibus, im\textbf{plé}bit ru\textbf{í}nas : \GreStar\ conquassábit cápita in \textbf{ter}ra mul\textbf{tó}rum.

De torrénte in \textbf{vi}a \textbf{bi}bet : \GreStar\ proptérea exal\textbf{tá}bit \textbf{ca}put.

Glória \textbf{Pa}tri, et \textbf{Fí}lio, \GreStar\ et Spi\textbf{rí}tui \textbf{Sanc}to.

Sicut erat in princípio, et \textbf{nunc}, et \textbf{sem}per, \GreStar\ et in sǽcula sæcu\textbf{ló}rum. \textbf{A}men.

\end{psalmtext}

\pend\pstart

\gregorioscore[a]{LatinVersion/Dominica/DomAnt1-02.gtex}

\pend\pstart

%% DOMINICA 2 ANT.

\greannotation{2 {\upshape\Abar} IV g}
\gregorioscore[a]{LatinVersion/Dominica/DomAnt2-01.gtex}

\pend\pstart

\redTitle{Psalmus 110.}

\pend\pstart

\gregorioscore[a]{LatinVersion/Dominica/DomFL2.gtex}

\pend\pstart

\begin{psalmtext}
Magna ó\emph{pera} \textbf{Dó}mini : \GreStar\ exquisíta in omnes voluntátes \textbf{e}jus.

Conféssio et magnificéntia \emph{opus} \textbf{e}jus : \GreStar\ et ju\libertineGlyph{s_t}ítia ejus manet in sǽculum \textbf{sǽ}culi.

Memóriam fécit mirabílium suórum, \GreDagger\ miséricors et mise\emph{rátor} \textbf{Dó}minus : \GreStar\ escam dedit timénti\textbf{bus} se.

Memor erit in sǽculum te\libertineGlyph{s_t}a\emph{ménti} \textbf{su}i : \GreStar\ virtútem óperum suórum annuntiábit pópulo \textbf{su}o :

Ut det illis hæredi\emph{tátem} \textbf{gén}tium : \GreStar\ ópera mánuum ejus véritas et ju\textbf{dí}cium.

Fidélia ómnia mandáta ejus : \GreDagger\ confirmáta in sǽ\emph{culum} \textbf{sǽ}culi : \GreStar\ fa\libertineGlyph{c_t}a in veritáte et æqui\textbf{tá}te.

Redemptiónem misit pó\emph{pulo} \textbf{su}o : \GreStar\ mandávit in ætérnum te\libertineGlyph{s_t}améntum \textbf{su}um.

\textcolor{benred8}{\emph{Fit reverentia :}} San\libertineGlyph{c_t}um et terríbile \emph{nomen} \textbf{e}jus : \GreStar\ inítium sapiéntiae timor \textbf{Dó}mini.

Intellé\libertineGlyph{c_t}us bonus ómnibus facién\emph{tibus} \textbf{e}um : \GreStar\ laudátio ejus manet in sǽculum \textbf{sǽ}culi.

Glória Pa\emph{tri, et} \textbf{Fí}lio, \GreStar\ et Spirítui \textbf{Sanc}to.

Sicut erat in princípio, et \emph{nunc, et} \textbf{sem}per, \GreStar\ et in sǽcula sæculórum. \textbf{A}men.

\end{psalmtext}

\pend\pstart

\gregorioscore[a]{LatinVersion/Dominica/DomAnt2-02.gtex}

\pend\pstart

%% DOMINICA 3 ANT.

\greannotation{3 {\upshape\Abar} IV a}
\gregorioscore[a]{LatinVersion/Dominica/DomAnt3-01.gtex}

\pend\pstart

\redTitle{Psalmus 111.}

\pend\pstart

\gregorioscore[a]{LatinVersion/Dominica/DomFL3.gtex}

\pend\pstart

\begin{psalmtext}
Potens in terra erit \emph{semen} \textbf{e}jus : \GreStar\ generátio re\libertineGlyph{c_t}órum \emph{benedi}\textbf{cé}tur.

Glória et divítiae in \emph{domo} \textbf{e}jus : \GreStar\ et ju\libertineGlyph{s_t}ítia ejus manet in \emph{sǽculum} \textbf{sǽ}culi.

Exórtum e\libertineGlyph{s_t} in ténebris \emph{lumen} \textbf{rec}tis : \GreStar\ miséricors, et mise\emph{rátor, et} \textbf{jus}tus.

Jucúndus homo qui miserétur et cómmodat, \GreDagger\ dispónet sermónes suos \emph{in ju}\textbf{dí}cio : \GreStar\ quia in ætérnum \emph{non commo}\textbf{vé}bitur.

In memória ætérna \emph{erit} \textbf{jus}tus : \GreStar\ ab auditióne ma\emph{la non ti}\textbf{mé}bit.

Parátum cor ejus speráre in Dómino, \GreDagger\ confirmátum \emph{e\libertineGlyph{s_t} cor} \textbf{e}jus : \GreStar\ non commovébitur donec despíciat i\emph{nimícos} \textbf{su}os.

Dispérsit, dedit paupéribus : \GreDagger\ ju\libertineGlyph{s_t}ítia ejus manet in sǽ\emph{culum} \textbf{sǽ}culi : \GreStar\ cornu ejus exaltá\emph{bitur in} \textbf{gló}ria.

Peccátor vidébit, et irascétur, \GreDagger\ déntibus suis fremet \emph{et ta}\textbf{bé}scet : \GreStar\ desidérium pecca\emph{tórum pe}\textbf{rí}bit.

Glória Pa\emph{tri, et} \textbf{Fí}lio, \GreStar\ et Sp\emph{irítui} \textbf{Sanc}to.

Sicut erat in princípio, et \emph{nunc, et} \textbf{sem}per, \GreStar\ et in sǽcula sæ\emph{culórum}. \textbf{A}men.

\end{psalmtext}

\pend\pstart

\gregorioscore[a]{LatinVersion/Dominica/DomAnt3-02.gtex}

\pend\pstart

%% DOMINICA 4 ANT.

\vspace{2mm}

\greannotation{4 {\upshape\Abar} VII c}
\gregorioscore[a]{LatinVersion/Dominica/DomAnt4-01.gtex}

\pend\pstart

\redTitle{Psalmus 112.}

\pend\pstart

\gregorioscore[a]{LatinVersion/Dominica/DomFL4.gtex}

\pend\pstart

\begin{psalmtext}
\textcolor{benred8}{\emph{Fit reverentia :}} Sit nomen Dómini \textbf{be}ne\textbf{díc}tum, \GreStar\ ex hoc nunc, et \textbf{us}que in \textbf{sǽ}culum.

A solis ortu usque \textbf{ad} oc\textbf{cá}sum, \GreStar\ laudábile \textbf{no}men \textbf{Dó}mini.

Excélsus super omnes \textbf{gen}tes \textbf{Dó}minus, \GreStar\ et super cælos \textbf{gló}ria \textbf{e}jus.

Quis sicut Dóminus Deus no\libertineGlyph{s_t}er, qui in \textbf{al}tis \textbf{há}bitat, \GreStar\ et humília réspicit in cælo \textbf{et} in \textbf{ter}ra?

Súscitans a \textbf{ter}ra \textbf{í}nopem, \GreStar\ et de \libertineGlyph{s_t}ércore \textbf{é}rigens \textbf{páu}perem :

Ut cóllocet eum \textbf{cum} prin\textbf{cí}pibus, \GreStar\ cum princípibus \textbf{pó}puli \textbf{su}i.

Qui habitáre facit \libertineGlyph{s_t}éri\textbf{lem} in \textbf{do}mo, \GreStar\ matrem fili\textbf{ó}rum lae\textbf{tán}tem.

Glória \textbf{Pa}tri, et \textbf{Fí}lio, \GreStar\ et Spi\textbf{rí}tui \textbf{Sanc}to.

Sicut erat in princípio, et \textbf{nunc}, et \textbf{sem}per, \GreStar\ et in sǽcula sæcu\textbf{ló}rum. \textbf{A}men.

\end{psalmtext}

\pend\pstart

\gregorioscore[a]{LatinVersion/Dominica/DomAnt4-02.gtex}

\pend\pstart

%% DOMINICA 5 ANT.

\greannotation{5 {\upshape\Abar} T. Per.}
\gregorioscore[a]{LatinVersion/Dominica/DomAnt5-01.gtex}

\pend\pstart

\redTitle{Psalmus 113.}

\pend\pstart

\gregorioscore[a]{LatinVersion/Dominica/DomFL5.gtex}

\pend\pstart

\begin{psalmtext}
Fa\libertineGlyph{c_t}a e\libertineGlyph{s_t} Judǽa san\libertineGlyph{c_t}ifi\emph{cátio} \textbf{e}jus, \GreStar\ Israel potés\emph{tas} \textbf{e}jus.

Mare \emph{vidit, et} \textbf{fu}git : \GreStar\ Jordánis convérsus e\libertineGlyph{s_t} \emph{re}\textbf{trór}sum.

Montes exsultavé\emph{runt ut a}\textbf{rí}etes : \GreStar\ et colles sicut a\emph{gni} \textbf{ó}vium.

Quid e\libertineGlyph{s_t} tibi ma\emph{re quod fu}\textbf{gís}ti? \GreStar\ et tu Jordánis, quia convérsus es \emph{re}\textbf{trór}sum.

Montes exsultá\libertineGlyph{s_t}is \emph{sicut a}\textbf{rí}etes, \GreStar\ et colles sicut a\emph{gni} \textbf{ó}vium?

A fácie Dómini \emph{mota e\libertineGlyph{s_t}} \textbf{ter}ra, \GreStar\ a fácie De\emph{i} \textbf{Ja}cob : 

Qui convértit petram in \emph{\libertineGlyph{s_t}agna a}\textbf{quá}rum, \GreStar\ et rupem in fontes \emph{a}\textbf{quá}rum.

Non nobis Dó\emph{mine, non} \textbf{no}bis : \GreStar\ sed nómini tuo \emph{da} \textbf{gló}riam.

Super misericórdia tua et ve\emph{ritáte} \textbf{tu}a : \GreStar\ nequándo dicant gentes : Ubi e\libertineGlyph{s_t} Deus \emph{e}\textbf{ó}rum?

Deus autem \emph{no\libertineGlyph{s_t}er in} \textbf{cæ}lo : \GreStar\ ómnia quæcúmque vólu\emph{it}, \textbf{fe}cit.

Simulácra géntium ar\emph{géntum et} \textbf{au}rum, \GreStar\ ópera mánu\emph{um} \textbf{hó}minum.

Os habent, \emph{et non lo}\textbf{quén}tur : \GreStar\ óculos habent, et non \emph{vi}\textbf{dé}bunt.

Aures ha\emph{bent, et non} \textbf{áu}dient : \GreStar\ nares habent, et non o\emph{do}\textbf{rá}bunt.

Manus habent, et non palpábunt : \GreDagger\ pedes habent, et \emph{non ambu}\textbf{lá}bunt : \GreStar\ non clamábunt in gúttu\emph{re} \textbf{su}o.

Símiles illis fiant qui \emph{fáciunt} \textbf{e}a : \GreStar\ et omnes qui confídunt \emph{in} \textbf{e}is.

Domus Israel spe\emph{rávit in} \textbf{Dó}mino : \GreStar\ adjútor eórum et proté\libertineGlyph{c_t}or \emph{e}\textbf{ó}rum e\libertineGlyph{s_t}.

Domus Aaron spe\emph{rávit in} \textbf{Dó}mino : \GreStar\ adjútor eórum et proté\libertineGlyph{c_t}or \emph{e}\textbf{ó}rum e\libertineGlyph{s_t}.

Qui timent Dóminum spera\emph{vérunt in} \textbf{Dó}mino : \GreStar\ adjútor eórum et proté\libertineGlyph{c_t}or \emph{e}\textbf{ó}rum e\libertineGlyph{s_t}.

Dóminus me\emph{mor fuit} \textbf{nos}tri : \GreStar\ et benedí\emph{xit} \textbf{no}bis.

Benedíxit \emph{dómui} \textbf{Is}rael : \GreStar\ benedíxit dómu\emph{i} \textbf{A}aron.

Benedíxit ómnibus \emph{qui timent} \textbf{Dó}minum \GreStar\ pusíllis cum \emph{ma}\textbf{jór}ibus.

Adjíciat \emph{Dóminus} \textbf{su}per vos : \GreStar\ super vos, et super fíli\emph{os} \textbf{ves}tros.

Benedíc\emph{ti vos a} \textbf{Dó}mino, \GreStar\ qui fecit cælum \emph{et} \textbf{ter}ram.

Cæ\emph{lum cæli} \textbf{Dó}mino : \GreStar\ terram autem dedit fíli\emph{is} \textbf{hó}minum.

Non mórtui lau\emph{dábunt te} \textbf{Dó}mine : \GreStar\ neque omnes qui descéndunt in \emph{in}\textbf{fér}num.

Sed nos qui vívimus bene\emph{dícimus} \textbf{Dó}mino, \GreStar\ ex hoc nunc et usque \emph{in} \textbf{sǽ}culum.

Glória \emph{Patri, et} \textbf{Fí}lio, \GreStar\ et Spirítu\emph{i} \textbf{Sanc}to.

Sicut erat in princípio, \emph{et nunc, et} \textbf{sem}per, \GreStar\ et in sǽcula sæculó\emph{rum}. \textbf{A}men.

\end{psalmtext}

\pend\pstart

\gregorioscore[a]{LatinVersion/Dominica/DomAnt5-02.gtex}

\pend\pstart

\begin{rubric}
Sequens Capitulum \emph{\textcolor{black}{Bened\'{i}\libertineGlyph{c_t}us Deus}} canitur a Dominica II po\libertineGlyph{s_t}  Epiphaniam usque ad Septuagesimam, et a Dominica III po\libertineGlyph{s_t} Penteco\libertineGlyph{s_t}en usque ad Adventum tantum. Hymnus vero canitur in iisdem Dominicis po\libertineGlyph{s_t} Penteco\libertineGlyph{s_t}en et Epiphaniam, etiam usque ad Dominicam I Quadragesim\ae .

\end{rubric}

\pend\pstart

% DOMINICA CAPITULUM

\redTitle{Capitulum.\capitulumSpace \emph{2 Cor. 1, 3--4.}}

\label{DominicaCapitulum}

\pend\pstart

\vspace{-5mm}

\gregorioscore[a]{LatinVersion/CapitulumBenedictus.gtex}

\color{black}

\pend\pstart

% DOMINICA HYMNUS HIEME

\vspace{4mm}

\redTitle{Hymnus.\\Tonus in Hieme.}

\pend\pstart

\begin{rubric}
Sequens tonus canitur in Dominicis po\libertineGlyph{s_t} Epiphaniam a die 14 Januarii usque ad Dominicam Quinquagesim\ae\ inclusive, et a Dominica proximiori Kalendis O\libertineGlyph{c_t}obris scilicet a die 28 Septembris usque ad Dominicam ultimam po\libertineGlyph{s_t} Penteco\libertineGlyph{s_t}en.

\end{rubric}

\pend\pstart

\greannotation{IV}
\gregorioscore[a]{LatinVersion/Dominica/DomHym-Hieme.gtex}

\pend\pstart

\begin{rubric}
Quando fit Commemoratio B. M. V., canitur doxologia \emph{\textcolor{black}{Gl\'{o}ria tibi, D\'{o}mine, Qui natus es de V\'{i}rgine, Cum Patre \emph{et} Sancto Sp\'{i}ritu, In sempit\'{e}rna s\'{\ae}cula}}, sed non mutatur tonus hymni.

\end{rubric}

\pend\pstart

{\gresetinitiallines{0}
\gregorioscore[a]{LatinVersion/DirigaturDomine.gtex}

}

\begin{response}
\hspace{\sicutIncensumIndent}\Rbar\ Sicut \hspace{1.1mm} inc\'{e}nsum \hspace{1.1mm} in \hspace{1.1mm} consp\'{e}\libertineGlyph{c_t}u \hspace{1.2mm} tu- o.

\end{response}

\pend\pstart

% DOMINICA HYMNUS AESTATE

\redTitle{Tonus in \AE \libertineGlyph{s_t}ate.}

\pend\pstart

\begin{rubric}
Sequens tonus canitur in Dominica IV et reliquis Dominicis po\libertineGlyph{s_t} Penteco\libertineGlyph{s_t}en usque ad Dominicam proximiorem Kalendis O\libertineGlyph{c_t}obris id e\libertineGlyph{s_t} ad diem 27 Septembris inclusive occurrentibus.

\end{rubric}

\pend\pstart

\greannotation{VIII}
\gregorioscore[a]{LatinVersion/Dominica/DomHym-Aestate.gtex}

\pend\pstart

\begin{response}
\Vbar\ Dirig\'{a}tur D\'{o}mine or\'{a}tio mea.\\
\Rbar\ Sicut inc\'{e}nsum in consp\'{e}\libertineGlyph{c_t}u tuo.

\end{response}

\pend\pstart

% DOMINICA HYMNUS AD LIB 1

\redTitle{Tonus ad libitum I.}

\pend\pstart

\greannotation{VIII}
\gregorioscore[a]{LatinVersion/Dominica/DomHym-AdLib1.gtex}

\pend\pstart

\begin{response}
\Vbar\ Dirig\'{a}tur D\'{o}mine or\'{a}tio mea.\\
\Rbar\ Sicut inc\'{e}nsum in consp\'{e}\libertineGlyph{c_t}u tuo.

\end{response}

\pend\pstart

% DOMINICA HYMNUS AD LIB 2

\redTitle{Tonus ad libitum II.}

\pend\pstart

\greannotation{I}
\gregorioscore[a]{LatinVersion/Dominica/DomHym-AdLib2.gtex}

\pend\pstart

\begin{response}
\Vbar\ Dirig\'{a}tur D\'{o}mine or\'{a}tio mea.\\
\Rbar\ Sicut inc\'{e}nsum in consp\'{e}\libertineGlyph{c_t}u tuo.

\end{response}

\pend\pstart

% DOMINICA MAGNIFICAT

\vspace{1mm}

\redTitle{Canticum Beat\ae\ Mari\ae\ Virginis.\capitulumSpace \emph{Luc. 1, 46--55.}}

\pend\pstart

\begin{rubric}
Canitur Antiphona propria.

\end{rubric}

\pend\pstart

\vspace{-2mm}

\begin{psalmtext}
\lettrine[lhang=0.70]{M}{a}gn\'{i}ficat \grealtcross\ \'{a}nima mea D\'{o}minum :
%Magn\'{i}ficat \GreStar\ \'{a}nima mea D\'{o}minum :

\hspace*{9.5 mm}Et exsult\'{a}vit sp\'{i}ritus meus \GreStar\ in Deo salut\'{a}ri meo.

Quia resp\'{e}xit humilit\'{a}tem anc\'{i}ll\ae\ su\ae\ : \GreStar\ ecce enim ex hoc be\'{a}tam me dicent omnes generati\'{o}nes.

Quia fecit mihi magna qui potens e\libertineGlyph{s_t} : \GreStar\ et san\libertineGlyph{c_t}um nomen ejus.

Et miseric\'{o}rdia ejus a prog\'{e}nie in prog\'{e}nies \GreStar\ tim\'{e}ntibus eum.

Fecit pot\'{e}ntiam in br\'{a}chio suo : \GreStar\ disp\'{e}rsit sup\'{e}rbos mente cordis sui.

Dep\'{o}suit pot\'{e}ntes de sede, \GreStar\ et exalt\'{a}vit h\'{u}miles.

Esuri\'{e}ntes impl\'{e}vit bonis : \GreStar\ et d\'{i}vites dim\'{i}sit in\'{a}nes.

Susc\'{e}pit Israel p\'{u}erum suum, \GreStar\ record\'{a}tus miseric\'{o}rdi\ae\ su\ae.

Sicut loc\'{u}tus e\libertineGlyph{s_t} ad patres nostros, \GreStar\ Abraham, et s\'{e}mini ejus in s\'{\ae}cula.

Glória Patri, et Fílio, \GreStar\ et Spirítui San\libertineGlyph{c_t}o.

Sicut erat in princípio, et nunc, et semper, \GreStar\ et in sǽcula sæculórum. Amen.

\end{psalmtext}

\pend\pstart

\begin{rubric}
Deinde repetitur Antiphona.

\end{rubric}

\pend\pstart

\vspace{-5mm}

\redTitle{Oratio.}

\pend\pstart

\vspace{1mm}

{
\grechangedim{beforeinitialshift}{0.25mm}{scalable}
\grechangedim{afterinitialshift}{0.25mm}{scalable}
\greannotation{\upshape\Vbar}
\gregorioscore[a]{LatinVersion/DomineExaudi-short-01.gtex}

}

\pend\pstart

\begin{rubric}
Canitur Oratio propria, secundum tonum ut habetur in p.~\pageref{sec:TonusOrationis}, et post eam, si occurrat eo die aliquod Fe\libertineGlyph{s_t}um Simplex, vel ad modum Simplicis recolendum, fit de eo commemoratio. Po\libertineGlyph{s_t}remo (~si id tempus requirit~) fiunt Commemorationes de San\libertineGlyph{c_t}a Maria, de San\libertineGlyph{c_t}o Joseph, de Apo\libertineGlyph{s_t}olis, et de Patrono Ecclesi\ae\ in ordine aliarum Commemorationum secundum illius dignitatem, et ultimo loco de Pace, ut infra in p.~\pageref{sec:Commem}. Post ultimam Orationem canitur :

\end{rubric}

\pend\pstart

{
\grechangedim{beforeinitialshift}{0.25mm}{scalable}
\grechangedim{afterinitialshift}{0.25mm}{scalable}
\greannotation{\upshape\Vbar}
\gregorioscore[a]{LatinVersion/DomineExaudi-short-01.gtex}

}

\pend\pstart

\vspace{1mm}

\begin{rubric}
Per annum :

\end{rubric}

\pend\pstart

\vspace{-3mm}

\greannotation{I}
\gregorioscore[a]{LatinVersion/Dominica/DomBenedicamus.gtex}

\pend\pstart

\begin{rubric}
Tempore Adventus et Quadragesim\ae\ :

\end{rubric}

\pend\pstart

\greannotation{IV}
\gregorioscore[a]{LatinVersion/Dominica/DomBenedicamusAdvLent.gtex}

\pend\pstart

\greannotation{\upshape\Vbar}
\gregorioscore[a]{LatinVersion/FideliumAnimae.gtex}

\pend\pstart

\begin{rubric}
Si po\libertineGlyph{s_t} Vesperas immediate sequatur Completorium, di\libertineGlyph{c_t}o \Vbar\ \emph{\textcolor{black}{Fid\'{e}lium \'{a}nim\ae}}, \libertineGlyph{s_t}atim incipitur \Vbar\ \emph{\textcolor{black}{Jube D\'{o}mine bened\'{i}cere}}. Secus autem, si tunc terminetur Officium, dicitur \emph{\textcolor{black}{Pater no\libertineGlyph{s_t}er}} totum secreto. %Oratione Dominica secreto recitata, dicitur : %, ut infra ad Completorium.

\end{rubric}

\pend\endnumbering
\end{Leftside}
%%%%%%%%%%%%%%%%%%%%%%%%%%%%%%%%%%%%%%%%%%%%%%%%%%%%%%
%%%%%%%%%%%%%%%%%%%%%%%%%%%%%%%%%%%%%%%%%%%%%%%%%%%%%%
\begin{Rightside}

\firstlinenum{10000}\linenumincrement{10000}\beginnumbering\pstart

% SUNDAY AT VESPERS

\begin{center}\begin{Huge}\textsc{\textcolor{benred8}{Sunday at Vespers}}\end{Huge}\end{center}

\pend\pstart

\vspace{2mm}

\begin{center}Our Father. Hail Mary.\end{center}

\pend\pstart

\grechangestyle{initial}{\fontsize{43}{43}\selectfont}
\greannotation{\upshape\Vbar}
\gregorioscore[a]{EnglishVersion/DeusInAdjutorium.gtex}

\pend\pstart

\begin{rubric}
The following is sung from Septuagesima Sunday until Easter in place of \emph{\textcolor{black}{Alleluia}} :

\end{rubric}

\gregorioscore[a]{EnglishVersion/LausTibi.gtex}

\pend\pstart

\vspace{2mm}

\begin{center}\begin{large}\textbf{\textsc{\textcolor{benred8}{1 Antiphon. VII c 2}}}\end{large}\end{center}

\vspace{2mm}

\pend\pstart

{\gresetinitiallines{2}
\grechangestyle{initial}{\fontsize{160}{160}\selectfont}
\gregorioscore[a]{EnglishVersion/Dominica/DomAnt1-01.gtex}

}

\pend\pstart

\redTitle{Psalm 109.}

\pend\pstart

\begin{psalmtext}
%The Lord \textbf{said} to \textbf{my} Lord: \GreStar\ Sit thou \textbf{at} my \textbf{right} hand:

%Until I \textbf{make} thy \textbf{e}nemies \GreStar\ thy \textbf{foot}stool.

The Lord will send forth the sceptre of thy power \pipe\ out of Sion: \GreStar\ rule thou in the \pipe\ midst \uline{of~thy}\hspace{2mm}\uline{ene}mies.

With thee is the principality in the day of thy strength: in the \pipe\ brightness \uline{of the} saints: \GreStar\ from the womb before the day star \pipe\ I begot thee.

The Lord hath sworn, and \pipe\ he will \uline{not re}pent: \GreStar\ Thou art a priest for ever according to the order \pipe\ of Mel\uline{chise}dech.

The Lord \pipe\ at thy right hand \GreStar\ hath broken kings in the \pipe\ \textbf{day} \uline{of his} wrath.

He shall judge among nations, \pipe\ he \uline{shall fill} ruins: \GreStar\ he shall crush the heads in the \pipe\ land of many.

He shall drink of the \pipe\ torrent \uline{in the} way: \GreStar\ therefore shall he \pipe\ \textbf{lift} \uline{up the} head.

Glory be to the \pipe\ Fa\uline{ther, and}\hspace{2mm}\uline{to the} Son, \GreStar\ and to the \pipe\ Holy Spirit.

As it was in the beginning, is now, and \pipe\ ever shall be, \GreStar\ world without \pipe\ \textbf{end}. Amen.

\end{psalmtext}

\pend\pstart

\gregorioscore[a]{EnglishVersion/Dominica/DomAnt1-02.gtex}

\pend\pstart

%% SUNDAY 2 ANT.

\greannotation{2 {\upshape\Abar} IV g}
\gregorioscore[a]{EnglishVersion/Dominica/DomAnt2-01.gtex}

\pend\pstart

\redTitle{Psalm 110.}

\pend\pstart

\gregorioscore[a]{EnglishVersion/Dominica/DomFL2.gtex}

\pend\pstart

\begin{psalmtext}
%I will praise thee, O Lord, \emph{with my} \textbf{whole} heart ; \GreStar\ in the council of the just, and in the congre\textbf{ga}tion.

Great are \pipe\ the \textbf{works} \uline{of the} Lord: \GreStar\ sought out according to all \pipe\ his wills.

His work is \pipe\ praise and magni\uline{ficence}: \GreStar\ and his justice continueth for ever and \pipe\ ever.

He hath made a remembrance of his wonderful works, \GreDagger\ being a mer- \pipe\ ciful and gra\uline{cious~Lord}:~\GreStar\ he hath given food to them that \pipe\ fear him.

He will be mindful for e- \pipe\ ver of his co\uline{venant}: \GreStar\ he will shew forth to his people the power of \pipe\ his works.

That he may give them the inheri- \pipe\ tance of the Gentiles: \GreStar\ the works of his hands are truth and \pipe\ judgment.

All his commandments are faithful: \GreDagger\ confirmed for \pipe\ ever and ever, \GreStar\ made in truth and \pipe\ e\uline{quity}.

He hath sent redemp- \pipe\ tion to his people: \GreStar\ he hath commanded his covenant for \pipe\ ever.

\textcolor{benred8}{\emph{A sign of reverence is made:}} Holy and terri- \pipe\ ble \textbf{is} his name : \GreStar\ the fear of the Lord is the beginning of \pipe\ wisdom.

A good understanding \pipe\ to all that do it: \GreStar\ his praise continueth for ever and \pipe\ ever.

Glory be to the Father, \pipe\ and \textbf{to} the Son, \GreStar\ and to the Holy \pipe\ Spirit.

As it was in the beginning, is now, \pipe\ and ever shall be, \GreStar\ world without end. \pipe\ Amen.

\end{psalmtext}

\pend\pstart

\gregorioscore[a]{EnglishVersion/Dominica/DomAnt2-02.gtex}

\pend\pstart

 %% SUNDAY 3 ANT.

\greannotation{3 {\upshape\Abar} IV a}
\gregorioscore[a]{EnglishVersion/Dominica/DomAnt3-01.gtex}

\pend\pstart

\redTitle{Psalm 111.}

\pend\pstart

\gregorioscore[a]{EnglishVersion/Dominica/DomFL3.gtex}


\pend\pstart

\begin{psalmtext}
%Blessed is the man that \emph{feareth} \textbf{the} Lord: \GreStar\ he shall delight exceedingly \emph{in his com}\textbf{mand}ments.

His seed shall be \pipe\ migh\uline{ty up}\textbf{on} earth: \GreStar\ the generation of the right- \pipe\ eous \textbf{shall} be \uline{blessed}.

Glory and wealth shall \pipe\ be \textbf{in} his house: \GreStar\ and his justice remaineth for \pipe\ \textbf{e}ver and \uline{ever}.

To the righteous a light is \pipe\ risen up in \uline{darkness}: \GreStar\ he is merciful, and com- \pipe\ passionate and just.

Acceptable is the man that sheweth mercy and lendeth: \GreDagger\ he shall or- \pipe\ der his words with \uline{judgment}: \GreStar\ because he shall not \pipe\ be \textbf{moved} for \uline{ever}.

The just shall be in ever- \pipe\ \textbf{last}ing re\underline{membrance}: \GreStar\ he shall not fear \pipe\ the \textbf{e}vil \uline{hearing}.

His heart is ready to hope in the Lord: \GreDagger\ \pipe\ his \textbf{heart} is \uline{strengthened}, \GreStar\ he shall not be moved until he look \pipe\ over his e\uline{nemies}.

He hath distributed, he hath given to the poor: \GreDagger\ his justice remaineth \pipe\ for ever and \uline{ever}: \GreStar\ his horn shall be ex- \pipe\ \textbf{al}ted in \uline{glory}.

The wicked shall see, and shall be angry, \GreDagger\ he shall gnash with his \pipe\ \uline{\textbf{teeth} and}\hspace{2mm}\uline{\textbf{pine} a}way: \GreStar\ the desire of the \pipe\ \textbf{wi}cked shall \uline{perish}.

Glory be to the Father, \pipe\ and \textbf{to} the Son, \GreStar\ and to the \pipe\ \textbf{Holy} \uline{Spirit}.

As it was in the beginning, is now, \pipe\ and ever shall be, \GreStar\ world \pipe\ \uline{without} \textbf{end}. Amen.

\end{psalmtext}

\pend\pstart

\gregorioscore[a]{EnglishVersion/Dominica/DomAnt3-02.gtex}

\pend\pstart

%% SUNDAY 4 ANT.

\vspace{2mm}

\greannotation{4 {\upshape\Abar} VII c}
\gregorioscore[a]{EnglishVersion/Dominica/DomAnt4-01.gtex}

\pend\pstart

\redTitle{Psalm 112.}

\pend\pstart

\gregorioscore[a]{EnglishVersion/Dominica/DomFL4.gtex}

\pend\pstart

\begin{psalmtext}
%Praise the \textbf{Lord}, ye \textbf{child}ren: \GreStar\ praise ye the \textbf{name} of \textbf{the} Lord.

\textcolor{benred8}{\emph{A sign of reverence is made:}} Blessed be the \pipe\ \textbf{name} \uline{of the} Lord, \GreStar\ from henceforth \pipe\ now \uline{and for} ever.

From the rising of the sun unto the going \pipe\ \textbf{down} \uline{of the} same, \GreStar\ the name of the Lord is \pipe\ \textbf{worth}\uline{y of} praise.

The Lord is high a- \pipe\ bove all nations; \GreStar\ and his glory a- \pipe\ bove the heavens.

Who is as the Lord our God, who \pipe\ \textbf{dwel}\uline{leth on} high: \GreStar\ and looketh down on the low things in \pipe\ heaven \uline{and in} earth?

Raising up the \pipe\ needy \uline{from the} earth, \GreStar\ and lifting up the poor \pipe\ out \uline{of the} dunghill:

That he may \pipe\ place \uline{him with} princes, \GreStar\ with the princes \pipe\ of his people.

Who maketh a barren woman to \pipe\ \textbf{dwell} \uline{in a} house, \GreStar\ the joyful \pipe\ mo\uline{ther of} children.

Glory be to the \pipe\ Fa\uline{ther, and}\hspace{2mm}\uline{to the} Son, \GreStar\ and to the \pipe\ Holy Spirit.

As it was in the beginning, is now, and \pipe\ ever shall be, \GreStar\ world without \pipe\ \textbf{end}. Amen.

\end{psalmtext}

\pend\pstart

\gregorioscore[a]{EnglishVersion/Dominica/DomAnt4-02.gtex}

\pend\pstart

%% SUNDAY 5 ANT.

\greannotation{5 {\upshape\Abar} T. Per.}
\gregorioscore[a]{EnglishVersion/Dominica/DomAnt5-01.gtex}

\pend\pstart

\redTitle{Psalm 113.}

\pend\pstart

\gregorioscore[a]{EnglishVersion/Dominica/DomFL5.gtex}

\pend\pstart

\begin{psalmtext}
%When Isra- \pipe\ el went out of Egypt, \GreStar\ the house of Jacob from a \pipe\ bar\uline{barous} people:

Judea was \pipe\ made his sanctuary, \GreStar\ Israel \pipe\ his dominion.

The \pipe\ \textbf{sea} saw and \textbf{fled}: \GreStar\ Jor- \pipe\ dan was turned back.

The \pipe\ mountains skipped like \textbf{rams}, \GreStar\ and the hills like the \pipe\ \textbf{lambs} \uline{of the} flock.

What ailed thee, O thou \pipe\ sea, that thou didst \textbf{flee}: \GreStar\ and thou, O Jordan, that \pipe\ thou wast turned back?

Ye mountains, \pipe\ that ye skipped like \textbf{rams}, \GreStar\ and ye hills, like \pipe\ \textbf{lambs} \uline{of the} flock?

At the presence of the \pipe\ Lord the earth was \textbf{moved}, \GreStar\ at the presence of the \pipe\ \uline{\textbf{God}~of} Jacob:

Who turned the \pipe\ rock \uline{into} pools of water, \GreStar\ and the stony hill into \pipe\ foun\uline{tains of} waters.

Not to us, O \pipe\ \textbf{Lord}, not to \textbf{us}; \GreStar\ but to thy \pipe\ \uline{\textbf{name} give} glory.

For thy mer- \pipe\ cy, and for thy truth's sake: \GreStar\ lest the Gentiles should say: \pipe\ \uline{\textbf{Where} is} their God?

But our \pipe\ \textbf{God} is in heaven: \GreStar\ he hath done all things whatso- \pipe\ ever he would.

The idols of the \pipe\ Gen\uline{tiles are} sil\uline{ver and} \textbf{gold}, \GreStar\ the \pipe\ works \uline{of the} \uline{hands of} men.

They have \pipe\ mouths and \textbf{speak not}: \GreStar\ they have \pipe\ eyes and see not.

They have \pipe\ ears and \textbf{hear not}: \GreStar\ they have \pipe\ no\uline{ses and} smell not.

They have hands and feel not: \GreDagger\ they have \pipe\ feet and \textbf{walk not}: \GreStar\ neither shall they \pipe\ cry~\uline{out~through} their throat.

Let them that make them be- \pipe\ come like unto \textbf{them}: \GreStar\ and all \pipe\ such as \uline{trust in} them.

The house of Israel hath \pipe\ \textbf{hoped} in the \textbf{Lord}: \GreStar\ he is their helper and \pipe\ their protector.

The house of Aaron hath \pipe\ \textbf{hoped} in the \textbf{Lord}: \GreStar\ he is their helper and \pipe\ their protector.

They that fear the Lord have \pipe\ \textbf{hoped} in the \textbf{Lord}: \GreStar\ he is their helper and \pipe\ their protector.

The Lord \pipe\ hath been mindful of us, \GreStar\ \pipe\ and hath blessed us.

He hath \pipe\ blessed the house of Is\uline{rael}: \GreStar\ he hath blessed the \pipe\ house of Aaron.

He hath blessed \pipe\ all that fear the \textbf{Lord}, \GreStar\ both \pipe\ little and great.

May the Lord add \pipe\ \textbf{bles}sings upon you: \GreStar\ upon you, and u- \pipe\ pon your children.

Blessed \pipe\ be you of the \textbf{Lord}, \GreStar\ who made \pipe\ heaven and earth.

The heaven of \pipe\ heaven is the \textbf{Lord's}: \GreStar\ but the earth he has given to the \pipe\ children of men.

The dead shall not \pipe\ \textbf{praise} thee, O \textbf{Lord}: \GreStar\ nor any of them that \pipe\ go down to hell.

But we that \pipe\ \textbf{live} bless the \textbf{Lord}: \GreStar\ from this time \pipe\ now \uline{and for} ever.

Glory be to the \pipe\ Fa\uline{ther, and} to the \textbf{Son}, \GreStar\ and to the \pipe\ Holy Spirit.

As it was in the beginning, is \pipe\ now, and ever shall be, \GreStar\ world with- \pipe\ out end. Amen.

\end{psalmtext}

\pend\pstart

\gregorioscore[a]{EnglishVersion/Dominica/DomAnt5-02.gtex}

\pend\pstart

\begin{rubric}
The following Chapter \emph{\textcolor{black}{Blessed be the God}} is only sung from the Second Sunday after Epiphany until Septuagesima Sunday and from the Third Sunday after Pentecost until Advent. The Hymn is sung on those same Sundays, but also continues until the First Sunday of Lent.

\end{rubric}

\pend\pstart

% SUNDAY CHAPTER

\redTitle{Chapter.\capitulumSpace \emph{2 Cor. 1, 3--4.}}

\label{SundayChapter}

\pend\pstart

\vspace{-5mm}

\gregorioscore[a]{EnglishVersion/CapitulumBenedictus.gtex}

\color{black}

\pend\pstart

% SUNDAY WINTER HYMN

\vspace{4mm}

\redTitle{Hymn.\\Winter Tone.}

\pend\pstart

\begin{rubric}
The following tone is sung on Sundays after Epiphany between and including the 14\textsuperscript{th} of January and Quinqugesima Sunday. It is also sung from the Sunday nearest to the 1\textsuperscript{st} of October (the first Sunday on or after the 28\textsuperscript{th} of September) until the last Sunday after Pentecost.

\end{rubric}

\pend\pstart

\greannotation{IV}
\gregorioscore[a]{EnglishVersion/Dominica/DomHym-Hieme.gtex}

\pend\pstart

\begin{rubric}
When a Commemoration is made of the Blessed Virgin Mary, a different Doxology is sung for the Hymn, namely, \emph{\textcolor{black}{O Jesu, born of Virgin bright, Immortal glory be to thee; Praise to the Father infinite, And Holy Ghost eternally}}. This is sung to the same tone.

\end{rubric}

\pend\pstart

{\gresetinitiallines{0}
\gregorioscore[a]{EnglishVersion/DirigaturDomine.gtex}

}

\begin{response}
\hspace{\sicutIncensumIndent}\Rbar\ As \hspace{1.1mm} incense \hspace{1.1mm} in \hspace{1.5mm} thy \hspace{1.2mm} sight.

\end{response}

\pend\pstart

% SUNDAY SUMMER HYMN

\redTitle{Summer Tone.}

\pend\pstart

\begin{rubric}
The following tone is sung from the Fourth Sunday after Pentecost until the Sunday nearest to the 1\textsuperscript{st} of October (any Sundays up to and including the 27\textsuperscript{th} of September).

\end{rubric}

\pend\pstart

\greannotation{VIII}
\gregorioscore[a]{EnglishVersion/Dominica/DomHym-Aestate.gtex}

\pend\pstart

\begin{response}
\Vbar\ O Lord, direct my prayer.\\
\Rbar\ As incense in thy sight.

\end{response}

\pend\pstart

% SUNDAY HYMN AD LIB 1

\redTitle{\emph{Ad libitum} Tone I.}

\pend\pstart

\greannotation{VIII}
\gregorioscore[a]{EnglishVersion/Dominica/DomHym-AdLib1.gtex}

\pend\pstart

\begin{response}
\Vbar\ O Lord, direct my prayer.\\
\Rbar\ As incense in thy sight.

\end{response}

\pend\pstart

% SUNDAY HYMN AD LIB 2

\redTitle{\emph{Ad libitum} Tone II.}

\pend\pstart

\greannotation{I}
\gregorioscore[a]{EnglishVersion/Dominica/DomHym-AdLib2.gtex}

\pend\pstart

\begin{response}
\Vbar\ O Lord, direct my prayer.\\
\Rbar\ As incense in thy sight.

\end{response}

\pend\pstart

% SUNDAY MAGNIFICAT

\vspace{1mm}

\redTitle{Canticle of the Blessed Virgin Mary.\capitulumSpace \emph{Luke 1, 46--55.}}

\pend\pstart

\begin{rubric}
The antiphon for the Sunday is sung.

\end{rubric}

\pend\pstart

\vspace{-2mm}

\begin{psalmtext}
\lettrine[lhang=0.70]{M}{y} soul \grealtcross\ doth magnify the Lord.

\hspace*{9.5 mm}And my spirit hath rejoiced \GreStar\ in God my Saviour.

Because he hath regarded the humility of his handmaid; \GreStar\ for behold from henceforth all generations shall call me blessed.

Because he that is mighty, hath done great things to me; \GreStar\ and holy is his name.

And his mercy is from generation unto generations, \GreStar\ to them that fear him.

He hath shewed might in his arm: \GreStar\ he hath scattered the proud in the conceit of their heart.

He hath put down the mighty from their seat, \GreStar\ and hath exalted the humble.

He hath filled the hungry with good things; \GreStar\ and the rich he hath sent empty away.

He hath received Israel his servant, \GreStar\ being mindful of his mercy:

As he spoke to our fathers, \GreStar\ to Abraham and to his seed for ever.

Glory be to the Father, and to the Son, \GreStar\ and to the Holy Spirit.

As it was in the beginning, is now, and ever shall be, \GreStar\ world without end. Amen.

\end{psalmtext}

\pend\pstart

\begin{rubric}
The Antiphon is now repeated.

\end{rubric}

\pend\pstart

\vspace{-5mm}

\redTitle{Oration.}

\pend\pstart

\vspace{1mm}

{
\grechangedim{beforeinitialshift}{0.25mm}{scalable}
\grechangedim{afterinitialshift}{0.25mm}{scalable}
\greannotation{\upshape\Vbar}
\gregorioscore[a]{EnglishVersion/DomineExaudi-short-01.gtex}

}

\pend\pstart

\begin{rubric}
The Oration for the Sunday is sung to the tone given on p.~\pageref{sec:TonusOrationis}, and after that if any Simple Feast falls on the same day, its commemoration is sung next. Finally, the Commorations on p.~\pageref{sec:Commem} are sung, with the Commemorations of Saint Mary, Saint Joseph, the Apostles, and the Patron of the Church in order of their dignity and the Commemoration of Peace taking last place. After this final Oration the following is repeated:

\end{rubric}

\pend\pstart

{
\grechangedim{beforeinitialshift}{0.25mm}{scalable}
\grechangedim{afterinitialshift}{0.25mm}{scalable}
\greannotation{\upshape\Vbar}
\gregorioscore[a]{EnglishVersion/DomineExaudi-short-01.gtex}

}

\pend\pstart

\begin{rubric}
Throughout the year:

\end{rubric}

\pend\pstart

\vspace{-2mm}

\greannotation{I}
\gregorioscore[a]{EnglishVersion/Dominica/DomBenedicamus.gtex}

\pend\pstart

\begin{rubric}
During Advent and Lent:

\end{rubric}

\pend\pstart

\greannotation{IV}
\gregorioscore[a]{EnglishVersion/Dominica/DomBenedicamusAdvLent.gtex}

\pend\pstart

\greannotation{\upshape\Vbar}
\gregorioscore[a]{EnglishVersion/FideliumAnimae.gtex}

\pend\pstart

\begin{rubric}
If Compline follows Vespers, \emph{\textcolor{black}{Grant, Lord, a blessing}} is sung straightaway after \emph{\textcolor{black}{May the souls of the faithful}} has finished. Otherwise, if the Office finishes here, a silent \emph{\textcolor{black}{Our Father}} is said.

\end{rubric}

\pend\endnumbering
\end{Rightside}
\end{pages}
\Pages
