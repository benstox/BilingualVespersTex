% File generated by gregorio 2.3
\gregoriotexapiversion{20130616}%
\begingregorioscore%
\grefirstlinebottomspace{2}{0}%
\greinitial{O}%
\grebeginnotes %
\gresetinitialclef{c}{4}{a}%
\gresyllable{}{}{}{1}{bl}{e}{8}{}{%
\greglyph{\char 1028}{d}{i}{0}{}{%
\grevepisemus{j}{18}%
}%
\greendofglyph{1}%
\greglyph{\char 23}{i}{h}{0}{}{}%
}%
%
\gresyllable{bl}{e}{st}{1}{Cr}{e}{0}{}{%
\greglyph{\char 6145}{h}{h}{8}{}{}%
}%
%
\gresyllable{Cr}{e}{}{0}{}{a}{0}{}{%
\greglyph{\char 17}{h}{j}{0}{}{}%
}%
\gresyllable{}{a}{}{0}{t}{o}{0}{}{%
\greglyph{\char 22}{j}{i}{0}{}{}%
\greendofglyph{9}%
\greglyph{\char 19}{i}{g}{6}{}{}%
\greendofglyph{3}%
\greglyph{\char 19}{h}{g}{6}{}{}%
}%
\gresyllable{t}{o}{r}{1}{}{o}{0}{}{%
\greglyph{\char 17}{g}{h}{0}{}{}%
}%
%
\gresyllable{}{o}{f}{1}{th}{e}{8}{}{%
\greglyph{\char 17}{h}{g}{0}{}{}%
}%
%
\gresyllable{th}{e}{}{1}{l}{i}{0}{}{%
\greglyph{\char 5121}{g}{e}{8}{}{}%
}%
%
\gresyllable{l}{i}{ght,}{1}{}{}{11}{}{%
\greglyph{\char 17}{e}{g}{0}{}{}%
\greendofglyph{1}%
\greglyph{\char 1025}{g}{d}{0}{}{%
\grepunctummora{h}{0}{0}{0}%
}%
}%
%
\grebarsyllable{}{}{}{1}{Wh}{o}{0}{}{%
\gredivisiominima{}%
}%
%
\gresyllable{Wh}{o}{}{1}{m}{a}{2}{}{%
\greglyph{\char 1028}{d}{g}{0}{}{}%
}%
%
\gresyllable{m}{a}{k'st}{1}{th}{e}{0}{}{%
\greglyph{\char 11275}{g}{g}{2}{}{}%
}%
%
\gresyllable{th}{e}{}{1}{d}{ay}{8}{}{%
\greglyph{\char 17}{g}{f}{0}{}{}%
}%
%
\gresyllable{d}{ay}{}{1}{w}{i}{0}{}{%
\greglyph{\char 6145}{f}{d}{8}{}{}%
}%
%
\gresyllable{w}{i}{th}{1}{r}{a}{0}{}{%
\greglyph{\char 1025}{d}{g}{0}{}{}%
}%
%
\gresyllable{r}{a}{}{0}{di}{a}{0}{}{%
\greglyph{\char 17}{g}{f}{0}{}{}%
}%
\gresyllable{di}{a}{nce}{1}{br}{i}{8}{}{%
\greglyph{\char 17}{f}{e}{0}{}{}%
}%
%
\gresyllable{br}{i}{ght,}{1}{}{}{11}{}{%
\greglyph{\char 5121}{e}{h}{8}{}{%
\greaugmentumduplex{d}{e}{1}%
}%
}%
%
\grebarsyllable{}{}{}{1}{}{A}{0}{}{%
\gredivisiominor{}%
}%
%
\gresyllable{}{A}{nd}{1}{}{o}{8}{}{%
\greglyph{\char 1025}{h}{h}{0}{}{}%
}%
%
\gresyllable{}{o}{'er}{1}{th}{e}{0}{}{%
\greglyph{\char 6145}{h}{h}{8}{}{}%
}%
%
\gresyllable{th}{e}{}{1}{f}{o}{0}{}{%
\greglyph{\char 1026}{h}{i}{0}{}{}%
}%
%
\gresyllable{f}{o}{r}{0}{m}{i}{0}{}{%
\greglyph{\char 23}{i}{h}{0}{}{}%
\greendofglyph{9}%
\greglyph{\char 19}{h}{h}{6}{}{}%
\greendofglyph{3}%
\greglyph{\char 19}{g}{h}{6}{}{}%
}%
\gresyllable{m}{i}{ng}{1}{w}{o}{8}{}{%
\greglyph{\char 17}{h}{g}{0}{}{}%
}%
%
\gresyllable{w}{o}{rld}{1}{d}{i}{0}{}{%
\greglyph{\char 5121}{g}{f}{8}{}{}%
\greendofglyph{0}%
\greglyph{\char 6145}{f}{d}{8}{}{}%
}%
%
\gresyllable{d}{i}{dst}{1}{c}{a}{0}{}{%
\greglyph{\char 17}{d}{h}{0}{}{}%
}%
%
\gresyllable{c}{a}{ll}{1}{}{}{11}{}{%
\greglyph{\char 17}{h}{h}{0}{}{%
\grepunctummora{h}{0}{0}{0}%
}%
}%
%
\grebarsyllable{}{}{}{1}{Th}{e}{20}{}{%
\gredivisiominima{}%
}%
%
\gresyllable{Th}{e}{}{1}{l}{i}{0}{}{%
\greflat{i}{0}%
\greglyph{\char 1025}{h}{h}{0}{}{}%
}%
%
\gresyllable{l}{i}{ght}{1}{fr}{o}{0}{}{%
\greglyph{\char 22}{h}{g}{0}{}{}%
\greendofglyph{9}%
\greglyph{\char 19}{g}{g}{6}{}{}%
\greendofglyph{3}%
\greglyph{\char 19}{f}{g}{6}{}{}%
}%
%
\gresyllable{fr}{o}{m}{1}{ch}{a}{0}{}{%
\greglyph{\char 17}{g}{g}{0}{}{}%
\greendofglyph{0}%
\greglyph{\char 5121}{g}{e}{8}{}{}%
}%
%
\gresyllable{ch}{a}{}{0}{}{o}{0}{}{%
\greglyph{\char 23}{e}{d}{0}{}{}%
\greendofglyph{9}%
\greglyph{\char 19}{d}{d}{6}{}{}%
\greendofglyph{3}%
\greglyph{\char 19}{c}{d}{6}{}{%
\grevepisemus{b}{12}%
}%
}%
\gresyllable{}{o}{s}{1}{f}{i}{0}{}{%
\greglyph{\char 17}{d}{e}{0}{}{}%
}%
%
\gresyllable{f}{i}{rst}{1}{}{o}{0}{}{%
\greglyph{\char 1025}{e}{d}{0}{}{}%
}%
%
\gresyllable{}{o}{f}{1}{}{a}{0}{}{%
\greglyph{\char 17}{d}{d}{0}{}{}%
}%
%
\gresyllable{}{a}{ll;}{1}{}{}{12}{}{%
\greglyph{\char 17}{d}{d}{0}{}{%
\grepunctummora{d}{0}{0}{0}%
}%
}%
%
\grebarsyllable{}{}{}{1}{Wh}{o}{0}{}{%
\gredivisiofinalis{}%
}%
%
\gresyllable{Wh}{o}{se}{1}{w}{i}{8}{}{%
\greglyph{\char 1028}{d}{i}{0}{}{%
\grevepisemus{j}{18}%
}%
\greendofglyph{1}%
\greglyph{\char 23}{i}{h}{0}{}{}%
}%
%
\gresyllable{w}{i}{s}{0}{d}{o}{0}{}{%
\greglyph{\char 6145}{h}{h}{8}{}{}%
}%
\gresyllable{d}{o}{m}{1}{j}{oi}{0}{}{%
\greglyph{\char 17}{h}{j}{0}{}{}%
}%
%
\gresyllable{j}{oi}{ned}{1}{}{i}{0}{}{%
\greglyph{\char 22}{j}{i}{0}{}{}%
\greendofglyph{9}%
\greglyph{\char 19}{i}{g}{6}{}{}%
\greendofglyph{3}%
\greglyph{\char 19}{h}{g}{6}{}{}%
}%
%
\gresyllable{}{i}{n}{1}{m}{ee}{0}{}{%
\greglyph{\char 17}{g}{h}{0}{}{}%
}%
%
\gresyllable{m}{ee}{t}{1}{}{a}{8}{}{%
\greglyph{\char 17}{h}{g}{0}{}{}%
}%
%
\gresyllable{}{a}{r}{0}{r}{ay}{0}{}{%
\greglyph{\char 5121}{g}{e}{8}{}{}%
}%
\gresyllable{r}{ay}{}{1}{}{}{11}{}{%
\greglyph{\char 17}{e}{g}{0}{}{}%
\greendofglyph{1}%
\greglyph{\char 1025}{g}{d}{0}{}{%
\grepunctummora{h}{0}{0}{0}%
}%
}%
%
\grebarsyllable{}{}{}{1}{Th}{e}{0}{}{%
\gredivisiominima{}%
}%
%
\gresyllable{Th}{e}{}{1}{m}{o}{2}{}{%
\greglyph{\char 1028}{d}{g}{0}{}{}%
}%
%
\gresyllable{m}{o}{rn}{1}{}{a}{0}{}{%
\greglyph{\char 11275}{g}{g}{2}{}{}%
}%
%
\gresyllable{}{a}{nd}{1}{}{e}{8}{}{%
\greglyph{\char 17}{g}{f}{0}{}{}%
}%
%
\gresyllable{}{e}{ve,}{1}{}{a}{0}{}{%
\greglyph{\char 6145}{f}{d}{8}{}{}%
}%
%
\gresyllable{}{a}{nd}{1}{n}{a}{0}{}{%
\greglyph{\char 1025}{d}{g}{0}{}{}%
}%
%
\gresyllable{n}{a}{med}{1}{th}{e}{0}{}{%
\greglyph{\char 17}{g}{f}{0}{}{}%
}%
%
\gresyllable{th}{e}{m}{1}{d}{ay}{8}{}{%
\greglyph{\char 17}{f}{e}{0}{}{}%
}%
%
\gresyllable{d}{ay}{:}{1}{}{}{11}{}{%
\greglyph{\char 5121}{e}{h}{8}{}{%
\greaugmentumduplex{d}{e}{1}%
}%
}%
%
\grebarsyllable{}{}{}{1}{N}{i}{0}{}{%
\gredivisiominor{}%
}%
%
\gresyllable{N}{i}{ght}{1}{c}{o}{8}{}{%
\greglyph{\char 1025}{h}{h}{0}{}{}%
}%
%
\gresyllable{c}{o}{mes}{1}{w}{i}{0}{}{%
\greglyph{\char 6145}{h}{h}{8}{}{}%
}%
%
\gresyllable{w}{i}{th}{1}{}{a}{0}{}{%
\greglyph{\char 1026}{h}{i}{0}{}{}%
}%
%
\gresyllable{}{a}{ll}{1}{}{i}{0}{}{%
\greglyph{\char 23}{i}{h}{0}{}{}%
\greendofglyph{9}%
\greglyph{\char 19}{h}{h}{6}{}{}%
\greendofglyph{3}%
\greglyph{\char 19}{g}{h}{6}{}{}%
}%
%
\gresyllable{}{i}{ts}{1}{d}{a}{8}{}{%
\greglyph{\char 17}{h}{g}{0}{}{}%
}%
%
\gresyllable{d}{a}{rk}{0}{l}{i}{0}{}{%
\greglyph{\char 5121}{g}{f}{8}{}{}%
\greendofglyph{0}%
\greglyph{\char 6145}{f}{d}{8}{}{}%
}%
\gresyllable{l}{i}{ng}{1}{f}{ea}{0}{}{%
\greglyph{\char 17}{d}{h}{0}{}{}%
}%
%
\gresyllable{f}{ea}{rs;}{1}{}{}{11}{}{%
\greglyph{\char 17}{h}{h}{0}{}{%
\grepunctummora{h}{0}{0}{0}%
}%
}%
%
\grebarsyllable{}{}{}{1}{R}{e}{20}{}{%
\gredivisiominima{}%
}%
%
\gresyllable{R}{e}{}{0}{g}{a}{0}{}{%
\greflat{i}{0}%
\greglyph{\char 1025}{h}{h}{0}{}{}%
}%
\gresyllable{g}{a}{rd}{1}{th}{y}{0}{}{%
\greglyph{\char 22}{h}{g}{0}{}{}%
\greendofglyph{9}%
\greglyph{\char 19}{g}{g}{6}{}{}%
\greendofglyph{3}%
\greglyph{\char 19}{f}{g}{6}{}{}%
}%
%
\gresyllable{th}{y}{}{1}{p}{eo}{0}{}{%
\greglyph{\char 17}{g}{g}{0}{}{}%
\greendofglyph{0}%
\greglyph{\char 5121}{g}{e}{8}{}{}%
}%
%
\gresyllable{p}{eo}{}{0}{pl}{e}{0}{}{%
\greglyph{\char 23}{e}{d}{0}{}{}%
\greendofglyph{9}%
\greglyph{\char 19}{d}{d}{6}{}{}%
\greendofglyph{3}%
\greglyph{\char 19}{c}{d}{6}{}{%
\grevepisemus{b}{12}%
}%
}%
\gresyllable{pl}{e}{'s}{1}{pr}{aye}{0}{}{%
\greglyph{\char 17}{d}{e}{0}{}{}%
}%
%
\gresyllable{pr}{aye}{rs}{1}{}{a}{0}{}{%
\greglyph{\char 1025}{e}{d}{0}{}{}%
}%
%
\gresyllable{}{a}{nd}{1}{t}{ea}{0}{}{%
\greglyph{\char 17}{d}{d}{0}{}{}%
}%
%
\gresyllable{t}{ea}{rs.}{1}{}{}{12}{}{%
\greglyph{\char 17}{d}{d}{0}{}{%
\grepunctummora{d}{0}{0}{0}%
}%
}%
%
\grebarsyllable{}{}{}{1}{L}{e}{0}{}{%
\gredivisiofinalis{}%
}%
%
\gresyllable{L}{e}{st,}{1}{s}{u}{8}{}{%
\greglyph{\char 1028}{d}{i}{0}{}{%
\grevepisemus{j}{18}%
}%
\greendofglyph{1}%
\greglyph{\char 23}{i}{h}{0}{}{}%
}%
%
\gresyllable{s}{u}{nk}{1}{}{i}{0}{}{%
\greglyph{\char 6145}{h}{h}{8}{}{}%
}%
%
\gresyllable{}{i}{n}{1}{s}{i}{0}{}{%
\greglyph{\char 17}{h}{j}{0}{}{}%
}%
%
\gresyllable{s}{i}{n,}{1}{}{a}{0}{}{%
\greglyph{\char 22}{j}{i}{0}{}{}%
\greendofglyph{9}%
\greglyph{\char 19}{i}{g}{6}{}{}%
\greendofglyph{3}%
\greglyph{\char 19}{h}{g}{6}{}{}%
}%
%
\gresyllable{}{a}{nd}{1}{wh}{e}{0}{}{%
\greglyph{\char 17}{g}{h}{0}{}{}%
}%
%
\gresyllable{wh}{e}{lmed}{1}{w}{i}{8}{}{%
\greglyph{\char 17}{h}{g}{0}{}{}%
}%
%
\gresyllable{w}{i}{th}{1}{str}{i}{0}{}{%
\greglyph{\char 5121}{g}{e}{8}{}{}%
}%
%
\gresyllable{str}{i}{fe,}{1}{}{}{11}{}{%
\greglyph{\char 17}{e}{g}{0}{}{}%
\greendofglyph{1}%
\greglyph{\char 1025}{g}{d}{0}{}{%
\grepunctummora{h}{0}{0}{0}%
}%
}%
%
\grebarsyllable{}{}{}{1}{Th}{ey}{0}{}{%
\gredivisiominima{}%
}%
%
\gresyllable{Th}{ey}{}{1}{l}{o}{2}{}{%
\greglyph{\char 1028}{d}{g}{0}{}{}%
}%
%
\gresyllable{l}{o}{se}{1}{th}{e}{0}{}{%
\greglyph{\char 11275}{g}{g}{2}{}{}%
}%
%
\gresyllable{th}{e}{}{1}{g}{i}{8}{}{%
\greglyph{\char 17}{g}{f}{0}{}{}%
}%
%
\gresyllable{g}{i}{ft}{1}{}{o}{0}{}{%
\greglyph{\char 6145}{f}{d}{8}{}{}%
}%
%
\gresyllable{}{o}{f}{1}{}{e}{0}{}{%
\greglyph{\char 1025}{d}{g}{0}{}{}%
}%
%
\gresyllable{}{e}{nd}{0}{l}{e}{0}{}{%
\greglyph{\char 17}{g}{f}{0}{}{}%
}%
\gresyllable{l}{e}{ss}{1}{l}{i}{8}{}{%
\greglyph{\char 17}{f}{e}{0}{}{}%
}%
%
\gresyllable{l}{i}{fe;}{1}{}{}{11}{}{%
\greglyph{\char 5121}{e}{h}{8}{}{%
\greaugmentumduplex{d}{e}{1}%
}%
}%
%
\grebarsyllable{}{}{}{1}{Wh}{i}{0}{}{%
\gredivisiominor{}%
}%
%
\gresyllable{Wh}{i}{le}{1}{th}{i}{8}{}{%
\greglyph{\char 1025}{h}{h}{0}{}{}%
}%
%
\gresyllable{th}{i}{nk}{0}{}{i}{0}{}{%
\greglyph{\char 6145}{h}{h}{8}{}{}%
}%
\gresyllable{}{i}{ng}{1}{b}{u}{0}{}{%
\greglyph{\char 1026}{h}{i}{0}{}{}%
}%
%
\gresyllable{b}{u}{t}{1}{th}{e}{0}{}{%
\greglyph{\char 23}{i}{h}{0}{}{}%
\greendofglyph{9}%
\greglyph{\char 19}{h}{h}{6}{}{}%
\greendofglyph{3}%
\greglyph{\char 19}{g}{h}{6}{}{}%
}%
%
\gresyllable{th}{e}{}{1}{th}{ou}{8}{}{%
\greglyph{\char 17}{h}{g}{0}{}{}%
}%
%
\gresyllable{th}{ou}{ghts}{1}{}{o}{0}{}{%
\greglyph{\char 5121}{g}{f}{8}{}{}%
\greendofglyph{0}%
\greglyph{\char 6145}{f}{d}{8}{}{}%
}%
%
\gresyllable{}{o}{f}{1}{t}{i}{0}{}{%
\greglyph{\char 17}{d}{h}{0}{}{}%
}%
%
\gresyllable{t}{i}{me,}{1}{}{}{11}{}{%
\greglyph{\char 17}{h}{h}{0}{}{%
\grepunctummora{h}{0}{0}{0}%
}%
}%
%
\grebarsyllable{}{}{}{1}{Th}{ey}{20}{}{%
\gredivisiominima{}%
}%
%
\gresyllable{Th}{ey}{}{1}{w}{ea}{0}{}{%
\greflat{i}{0}%
\greglyph{\char 1025}{h}{h}{0}{}{}%
}%
%
\gresyllable{w}{ea}{ve}{1}{n}{e}{0}{}{%
\greglyph{\char 22}{h}{g}{0}{}{}%
\greendofglyph{9}%
\greglyph{\char 19}{g}{g}{6}{}{}%
\greendofglyph{3}%
\greglyph{\char 19}{f}{g}{6}{}{}%
}%
%
\gresyllable{n}{e}{w}{1}{ch}{ai}{0}{}{%
\greglyph{\char 17}{g}{g}{0}{}{}%
\greendofglyph{0}%
\greglyph{\char 5121}{g}{e}{8}{}{}%
}%
%
\gresyllable{ch}{ai}{ns}{1}{}{o}{0}{}{%
\greglyph{\char 23}{e}{d}{0}{}{}%
\greendofglyph{9}%
\greglyph{\char 19}{d}{d}{6}{}{}%
\greendofglyph{3}%
\greglyph{\char 19}{c}{d}{6}{}{%
\grevepisemus{b}{12}%
}%
}%
%
\gresyllable{}{o}{f}{1}{w}{oe}{0}{}{%
\greglyph{\char 17}{d}{e}{0}{}{}%
}%
%
\gresyllable{w}{oe}{}{1}{}{a}{0}{}{%
\greglyph{\char 1025}{e}{d}{0}{}{}%
}%
%
\gresyllable{}{a}{nd}{1}{cr}{i}{0}{}{%
\greglyph{\char 17}{d}{d}{0}{}{}%
}%
%
\gresyllable{cr}{i}{me.}{1}{}{}{12}{}{%
\greglyph{\char 17}{d}{d}{0}{}{%
\grepunctummora{d}{0}{0}{0}%
}%
}%
%
\grebarsyllable{}{}{}{1}{B}{u}{0}{}{%
\gredivisiofinalis{}%
}%
%
\gresyllable{B}{u}{t}{1}{gr}{a}{8}{}{%
\greglyph{\char 1028}{d}{i}{0}{}{%
\grevepisemus{j}{18}%
}%
\greendofglyph{1}%
\greglyph{\char 23}{i}{h}{0}{}{}%
}%
%
\gresyllable{gr}{a}{nt}{1}{th}{e}{0}{}{%
\greglyph{\char 6145}{h}{h}{8}{}{}%
}%
%
\gresyllable{th}{e}{m}{1}{gr}{a}{0}{}{%
\greglyph{\char 17}{h}{j}{0}{}{}%
}%
%
\gresyllable{gr}{a}{ce}{1}{th}{a}{0}{}{%
\greglyph{\char 22}{j}{i}{0}{}{}%
\greendofglyph{9}%
\greglyph{\char 19}{i}{g}{6}{}{}%
\greendofglyph{3}%
\greglyph{\char 19}{h}{g}{6}{}{}%
}%
%
\gresyllable{th}{a}{t}{1}{th}{ey}{0}{}{%
\greglyph{\char 17}{g}{h}{0}{}{}%
}%
%
\gresyllable{th}{ey}{}{1}{m}{ay}{8}{}{%
\greglyph{\char 17}{h}{g}{0}{}{}%
}%
%
\gresyllable{m}{ay}{}{1}{str}{ai}{0}{}{%
\greglyph{\char 5121}{g}{e}{8}{}{}%
}%
%
\gresyllable{str}{ai}{n}{1}{}{}{11}{}{%
\greglyph{\char 17}{e}{g}{0}{}{}%
\greendofglyph{1}%
\greglyph{\char 1025}{g}{d}{0}{}{%
\grepunctummora{h}{0}{0}{0}%
}%
}%
%
\grebarsyllable{}{}{}{1}{Th}{e}{0}{}{%
\gredivisiominima{}%
}%
%
\gresyllable{Th}{e}{}{1}{h}{ea}{2}{}{%
\greglyph{\char 1028}{d}{g}{0}{}{}%
}%
%
\gresyllable{h}{ea}{ven}{0}{l}{y}{0}{}{%
\greglyph{\char 11275}{g}{g}{2}{}{}%
}%
\gresyllable{l}{y}{}{1}{g}{a}{8}{}{%
\greglyph{\char 17}{g}{f}{0}{}{}%
}%
%
\gresyllable{g}{a}{te}{1}{}{a}{0}{}{%
\greglyph{\char 6145}{f}{d}{8}{}{}%
}%
%
\gresyllable{}{a}{nd}{1}{pr}{i}{0}{}{%
\greglyph{\char 1025}{d}{g}{0}{}{}%
}%
%
\gresyllable{pr}{i}{ze}{1}{t}{o}{0}{}{%
\greglyph{\char 17}{g}{f}{0}{}{}%
}%
%
\gresyllable{t}{o}{}{1}{g}{ai}{8}{}{%
\greglyph{\char 17}{f}{e}{0}{}{}%
}%
%
\gresyllable{g}{ai}{n:}{1}{}{}{11}{}{%
\greglyph{\char 5121}{e}{h}{8}{}{%
\greaugmentumduplex{d}{e}{1}%
}%
}%
%
\grebarsyllable{}{}{}{1}{}{Ea}{0}{}{%
\gredivisiominor{}%
}%
%
\gresyllable{}{Ea}{ch}{1}{h}{a}{8}{}{%
\greglyph{\char 1025}{h}{h}{0}{}{}%
}%
%
\gresyllable{h}{a}{rm}{0}{f}{u}{0}{}{%
\greglyph{\char 6145}{h}{h}{8}{}{}%
}%
\gresyllable{f}{u}{l}{1}{l}{u}{0}{}{%
\greglyph{\char 1026}{h}{i}{0}{}{}%
}%
%
\gresyllable{l}{u}{re}{1}{}{a}{0}{}{%
\greglyph{\char 23}{i}{h}{0}{}{}%
\greendofglyph{9}%
\greglyph{\char 19}{h}{h}{6}{}{}%
\greendofglyph{3}%
\greglyph{\char 19}{g}{h}{6}{}{}%
}%
%
\gresyllable{}{a}{}{0}{s}{i}{8}{}{%
\greglyph{\char 17}{h}{g}{0}{}{}%
}%
\gresyllable{s}{i}{de}{1}{t}{o}{0}{}{%
\greglyph{\char 5121}{g}{f}{8}{}{}%
\greendofglyph{0}%
\greglyph{\char 6145}{f}{d}{8}{}{}%
}%
%
\gresyllable{t}{o}{}{1}{c}{a}{0}{}{%
\greglyph{\char 17}{d}{h}{0}{}{}%
}%
%
\gresyllable{c}{a}{st,}{1}{}{}{11}{}{%
\greglyph{\char 17}{h}{h}{0}{}{%
\grepunctummora{h}{0}{0}{0}%
}%
}%
%
\grebarsyllable{}{}{}{1}{}{A}{20}{}{%
\gredivisiominima{}%
}%
%
\gresyllable{}{A}{nd}{1}{p}{u}{0}{}{%
\greflat{i}{0}%
\greglyph{\char 1025}{h}{h}{0}{}{}%
}%
%
\gresyllable{p}{u}{rge}{1}{}{a}{0}{}{%
\greglyph{\char 22}{h}{g}{0}{}{}%
\greendofglyph{9}%
\greglyph{\char 19}{g}{g}{6}{}{}%
\greendofglyph{3}%
\greglyph{\char 19}{f}{g}{6}{}{}%
}%
%
\gresyllable{}{a}{}{0}{w}{ay}{0}{}{%
\greglyph{\char 17}{g}{g}{0}{}{}%
\greendofglyph{0}%
\greglyph{\char 5121}{g}{e}{8}{}{}%
}%
\gresyllable{w}{ay}{}{1}{}{ea}{0}{}{%
\greglyph{\char 23}{e}{d}{0}{}{}%
\greendofglyph{9}%
\greglyph{\char 19}{d}{d}{6}{}{}%
\greendofglyph{3}%
\greglyph{\char 19}{c}{d}{6}{}{%
\grevepisemus{b}{12}%
}%
}%
%
\gresyllable{}{ea}{ch}{1}{}{e}{0}{}{%
\greglyph{\char 17}{d}{e}{0}{}{}%
}%
%
\gresyllable{}{e}{r}{0}{r}{o}{0}{}{%
\greglyph{\char 1025}{e}{d}{0}{}{}%
}%
\gresyllable{r}{o}{r}{1}{p}{a}{0}{}{%
\greglyph{\char 17}{d}{d}{0}{}{}%
}%
%
\gresyllable{p}{a}{st.}{1}{}{}{12}{}{%
\greglyph{\char 17}{d}{d}{0}{}{%
\grepunctummora{d}{0}{0}{0}%
}%
}%
%
\grebarsyllable{}{}{}{1}{}{O}{0}{}{%
\gredivisiofinalis{}%
}%
%
\gresyllable{}{O}{}{1}{F}{a}{8}{}{%
\greglyph{\char 1028}{d}{i}{0}{}{%
\grevepisemus{j}{18}%
}%
\greendofglyph{1}%
\greglyph{\char 23}{i}{h}{0}{}{}%
}%
%
\gresyllable{F}{a}{}{0}{th}{e}{0}{}{%
\greglyph{\char 6145}{h}{h}{8}{}{}%
}%
\gresyllable{th}{e}{r,}{1}{th}{a}{0}{}{%
\greglyph{\char 17}{h}{j}{0}{}{}%
}%
%
\gresyllable{th}{a}{t}{1}{w}{e}{0}{}{%
\greglyph{\char 22}{j}{i}{0}{}{}%
\greendofglyph{9}%
\greglyph{\char 19}{i}{g}{6}{}{}%
\greendofglyph{3}%
\greglyph{\char 19}{h}{g}{6}{}{}%
}%
%
\gresyllable{w}{e}{}{1}{}{a}{0}{}{%
\greglyph{\char 17}{g}{h}{0}{}{}%
}%
%
\gresyllable{}{a}{sk}{1}{b}{e}{8}{}{%
\greglyph{\char 17}{h}{g}{0}{}{}%
}%
%
\gresyllable{b}{e}{}{1}{d}{o}{0}{}{%
\greglyph{\char 5121}{g}{e}{8}{}{}%
}%
%
\gresyllable{d}{o}{ne,}{1}{}{}{11}{}{%
\greglyph{\char 17}{e}{g}{0}{}{}%
\greendofglyph{1}%
\greglyph{\char 1025}{g}{d}{0}{}{%
\grepunctummora{h}{0}{0}{0}%
}%
}%
%
\grebarsyllable{}{}{}{1}{Thr}{ou}{0}{}{%
\gredivisiominima{}%
}%
%
\gresyllable{Thr}{ou}{gh}{1}{J}{e}{2}{}{%
\greglyph{\char 1028}{d}{g}{0}{}{}%
}%
%
\gresyllable{J}{e}{}{0}{s}{u}{0}{}{%
\greglyph{\char 11275}{g}{g}{2}{}{}%
}%
\gresyllable{s}{u}{s}{1}{Chr}{i}{8}{}{%
\greglyph{\char 17}{g}{f}{0}{}{}%
}%
%
\gresyllable{Chr}{i}{st,}{1}{th}{i}{0}{}{%
\greglyph{\char 6145}{f}{d}{8}{}{}%
}%
%
\gresyllable{th}{i}{ne}{1}{}{o}{0}{}{%
\greglyph{\char 1025}{d}{g}{0}{}{}%
}%
%
\gresyllable{}{o}{n}{0}{l}{y}{0}{}{%
\greglyph{\char 17}{g}{f}{0}{}{}%
}%
\gresyllable{l}{y}{}{1}{S}{o}{8}{}{%
\greglyph{\char 17}{f}{e}{0}{}{}%
}%
%
\gresyllable{S}{o}{n;}{1}{}{}{11}{}{%
\greglyph{\char 5121}{e}{h}{8}{}{%
\greaugmentumduplex{d}{e}{1}%
}%
}%
%
\grebarsyllable{}{}{}{1}{Wh}{o}{0}{}{%
\gredivisiominor{}%
}%
%
\gresyllable{Wh}{o}{,}{1}{w}{i}{8}{}{%
\greglyph{\char 1025}{h}{h}{0}{}{}%
}%
%
\gresyllable{w}{i}{th}{1}{th}{e}{0}{}{%
\greglyph{\char 6145}{h}{h}{8}{}{}%
}%
%
\gresyllable{th}{e}{}{1}{H}{o}{0}{}{%
\greglyph{\char 1026}{h}{i}{0}{}{}%
}%
%
\gresyllable{H}{o}{}{0}{l}{y}{0}{}{%
\greglyph{\char 23}{i}{h}{0}{}{}%
\greendofglyph{9}%
\greglyph{\char 19}{h}{h}{6}{}{}%
\greendofglyph{3}%
\greglyph{\char 19}{g}{h}{6}{}{}%
}%
\gresyllable{l}{y}{}{1}{Gh}{o}{8}{}{%
\greglyph{\char 17}{h}{g}{0}{}{}%
}%
%
\gresyllable{Gh}{o}{st}{1}{}{a}{0}{}{%
\greglyph{\char 5121}{g}{f}{8}{}{}%
\greendofglyph{0}%
\greglyph{\char 6145}{f}{d}{8}{}{}%
}%
%
\gresyllable{}{a}{nd}{1}{th}{ee}{0}{}{%
\greglyph{\char 17}{d}{h}{0}{}{}%
}%
%
\gresyllable{th}{ee}{,}{1}{}{}{11}{}{%
\greglyph{\char 17}{h}{h}{0}{}{%
\grepunctummora{h}{0}{0}{0}%
}%
}%
%
\grebarsyllable{}{}{}{1}{D}{o}{20}{}{%
\gredivisiominima{}%
}%
%
\gresyllable{D}{o}{th}{1}{l}{i}{0}{}{%
\greflat{i}{0}%
\greglyph{\char 1025}{h}{h}{0}{}{}%
}%
%
\gresyllable{l}{i}{ve}{1}{}{a}{0}{}{%
\greglyph{\char 22}{h}{g}{0}{}{}%
\greendofglyph{9}%
\greglyph{\char 19}{g}{g}{6}{}{}%
\greendofglyph{3}%
\greglyph{\char 19}{f}{g}{6}{}{}%
}%
%
\gresyllable{}{a}{nd}{1}{r}{ei}{0}{}{%
\greglyph{\char 17}{g}{g}{0}{}{}%
\greendofglyph{0}%
\greglyph{\char 5121}{g}{e}{8}{}{}%
}%
%
\gresyllable{r}{ei}{gn}{1}{}{e}{0}{}{%
\greglyph{\char 23}{e}{d}{0}{}{}%
\greendofglyph{9}%
\greglyph{\char 19}{d}{d}{6}{}{}%
\greendofglyph{3}%
\greglyph{\char 19}{c}{d}{6}{}{%
\grevepisemus{b}{12}%
}%
}%
%
\gresyllable{}{e}{}{0}{t}{e}{0}{}{%
\greglyph{\char 17}{d}{e}{0}{}{}%
}%
\gresyllable{t}{e}{r}{0}{n}{a}{0}{}{%
\greglyph{\char 1025}{e}{d}{0}{}{}%
}%
\gresyllable{n}{a}{l}{0}{l}{y}{0}{}{%
\greglyph{\char 17}{d}{d}{0}{}{}%
}%
\gresyllable{l}{y}{.}{1}{}{}{12}{}{%
\greglyph{\char 17}{d}{d}{0}{}{%
\grepunctummora{d}{0}{0}{0}%
}%
}%
%
\grebarsyllable{}{}{}{1}{}{A}{0}{}{%
\gredivisiofinalis{}%
}%
%
\gresyllable{}{A}{}{0}{m}{e}{0}{}{%
\greglyph{\char 15366}{d}{c}{0}{}{}%
}%
\gresyllable{m}{e}{n.}{1}{}{}{12}{}{%
\greglyph{\char 1025}{c}{g}{0}{}{%
\greaugmentumduplex{c}{d}{1}%
}%
}%
%
\grefinaldivisiofinalis{0}%
\endgregorioscore %
\endinput %
